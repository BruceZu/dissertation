\chapter{Safety proof and formal specification}
\label{appendix:correctness}

This appendix includes a formal specification and a proof of safety for
the basic Raft algorithm presented in Chapter~\ref{basicraft}. The
specification and proof are introduced in Chapter~\ref{correctness}.

The formal specification makes the information summarized in
Figure~\ref{fig:basicraft:cheatsheet} completely precise using the TLA+
specification language~\cite{Lamport:2002}. It serves as the subject of
the proof and is a useful reference for implementing Raft.

The proof shows that the specification preserves the State Machine
Safety property. The main idea of the proof is summarized in
Section~\ref{basicraft:safety:argument}, but the detailed proof is much
more precise. We found the proof useful in understanding Raft's safety
at a deeper level, and others may find value in this as well. However,
the proof is fairly long and difficult for humans to verify and
maintain; we believe it to be basically correct, but it might include
errors or omissions. At this scale, only a machine-checked proof could
definitively be error-free.

\section{Conventions}
The specification uses the syntax and semantics of the TLA+ language
version~2~\cite{Lamport:2002}.
The proof uses the same syntax and semantics but
with the following minor allowances for convenience:
\begin{itemize}
\item As in TLA+, $foo'$ has a specific meaning: the value of variable
$foo$ in the next state of the system.

\item Define $\langle index, term \rangle \in log$ iff \\
      $Len(log) \geq index \sland log[index].term = term$.

\item The symbol $\cat$ is used for concatenation of logs and entries.

\item Values in log entries are not included, since a
value is attached to a particular $\langle index, term \rangle$, and
those uniquely identify a log entry.
\end{itemize}

\section{Specification}

This section provides a complete, formal description of the Raft
algorithm.
A copy of the TLA+ source file can is available at~\cite{raft.tla}.

\tlatex
\setboolean{shading}{true}
 \@x{\makebox[0pt][r]{\scriptsize 1\hspace{1em}}}\moduleLeftDash\@xx{
 {\MODULE} raft}\moduleRightDash\@xx{}%
\@x{\makebox[0pt][r]{\scriptsize 2\hspace{1em}}}%
\@y{\@s{0}%
 This is the formal specification for the Raft consensus algorithm.
}%
\@xx{}%
\@x{\makebox[0pt][r]{\scriptsize 3\hspace{1em}}}%
\@y{\@s{0}%
 It was last modified on \ensuremath{July} 6, 2014.
}%
\@xx{}%
\@pvspace{8.0pt}%
 \@x{\makebox[0pt][r]{\scriptsize 5\hspace{1em}} {\EXTENDS} Naturals ,\,
 FiniteSets ,\, Sequences ,\, TLC}%
\@pvspace{8.0pt}%
\@x{\makebox[0pt][r]{\scriptsize 7\hspace{1em}}}%
\@y{\@s{0}%
 The set of server \ensuremath{IDs
}}%
\@xx{}%
\@x{\makebox[0pt][r]{\scriptsize 8\hspace{1em}} {\CONSTANTS} Server}%
\@pvspace{8.0pt}%
\@x{\makebox[0pt][r]{\scriptsize 10\hspace{1em}}}%
\@y{\@s{0}%
 The set of requests that can go into the \ensuremath{log
}}%
\@xx{}%
\@x{\makebox[0pt][r]{\scriptsize 11\hspace{1em}} {\CONSTANTS} Value}%
\@pvspace{8.0pt}%
\@x{\makebox[0pt][r]{\scriptsize 13\hspace{1em}}}%
\@y{\@s{0}%
 Server states.
}%
\@xx{}%
 \@x{\makebox[0pt][r]{\scriptsize 14\hspace{1em}} {\CONSTANTS} Follower ,\,
 Candidate ,\, Leader}%
\@pvspace{8.0pt}%
\@x{\makebox[0pt][r]{\scriptsize 16\hspace{1em}}}%
\@y{\@s{0}%
 A reserved value.
}%
\@xx{}%
\@x{\makebox[0pt][r]{\scriptsize 17\hspace{1em}} {\CONSTANTS} Nil}%
\@pvspace{8.0pt}%
\@x{\makebox[0pt][r]{\scriptsize 19\hspace{1em}}}%
\@y{\@s{0}%
 Message types:
}%
\@xx{}%
 \@x{\makebox[0pt][r]{\scriptsize 20\hspace{1em}} {\CONSTANTS}
 RequestVoteRequest ,\, RequestVoteResponse ,\,}%
 \@x{\makebox[0pt][r]{\scriptsize 21\hspace{1em}}\@s{59.95}
 AppendEntriesRequest ,\, AppendEntriesResponse}%
\@pvspace{8.0pt}%
\@x{\makebox[0pt][r]{\scriptsize 23\hspace{1em}}}\midbar\@xx{}%
\@x{\makebox[0pt][r]{\scriptsize 24\hspace{1em}}}%
\@y{\@s{0}%
 Global variables
}%
\@xx{}%
\@pvspace{8.0pt}%
\@x{\makebox[0pt][r]{\scriptsize 26\hspace{1em}}}%
\@y{\@s{0}%
 A bag of records representing requests and responses sent from one server
}%
\@xx{}%
\@x{\makebox[0pt][r]{\scriptsize 27\hspace{1em}}}%
\@y{\@s{0}%
 to another. \ensuremath{TLAPS} doesn\mbox{'}t support the Bags module, so
 this is a function
}%
\@xx{}%
\@x{\makebox[0pt][r]{\scriptsize 28\hspace{1em}}}%
\@y{\@s{0}%
 mapping Message to \ensuremath{Nat}.
}%
\@xx{}%
\@x{\makebox[0pt][r]{\scriptsize 29\hspace{1em}} {\VARIABLE} messages}%
\@pvspace{8.0pt}%
\@x{\makebox[0pt][r]{\scriptsize 31\hspace{1em}}}%
\@y{\@s{0}%
 A history variable used in the proof. This would not be present in an
}%
\@xx{}%
\@x{\makebox[0pt][r]{\scriptsize 32\hspace{1em}}}%
\@y{\@s{0}%
 implementation.
}%
\@xx{}%
\@x{\makebox[0pt][r]{\scriptsize 33\hspace{1em}}}%
\@y{\@s{0}%
 Keeps track of successful elections, including the initial logs of the
}%
\@xx{}%
\@x{\makebox[0pt][r]{\scriptsize 34\hspace{1em}}}%
\@y{\@s{0}%
 leader and voters\mbox{'} logs. Set of functions containing various things
 about
}%
\@xx{}%
\@x{\makebox[0pt][r]{\scriptsize 35\hspace{1em}}}%
\@y{\@s{0}%
 successful elections (see \ensuremath{BecomeLeader}).
}%
\@xx{}%
\@x{\makebox[0pt][r]{\scriptsize 36\hspace{1em}} {\VARIABLE} elections}%
\@pvspace{8.0pt}%
\@x{\makebox[0pt][r]{\scriptsize 38\hspace{1em}}}%
\@y{\@s{0}%
 A history variable used in the proof. This would not be present in an
}%
\@xx{}%
\@x{\makebox[0pt][r]{\scriptsize 39\hspace{1em}}}%
\@y{\@s{0}%
 implementation.
}%
\@xx{}%
\@x{\makebox[0pt][r]{\scriptsize 40\hspace{1em}}}%
\@y{\@s{0}%
 Keeps track of every \ensuremath{log} ever in the system (set of logs).
}%
\@xx{}%
\@x{\makebox[0pt][r]{\scriptsize 41\hspace{1em}} {\VARIABLE} allLogs}%
\@pvspace{8.0pt}%
\@x{\makebox[0pt][r]{\scriptsize 43\hspace{1em}}}\midbar\@xx{}%
\@x{\makebox[0pt][r]{\scriptsize 44\hspace{1em}}}%
\@y{\@s{0}%
 The following variables are all per server (functions with domain
 \ensuremath{Server}).
}%
\@xx{}%
\@pvspace{8.0pt}%
\@x{\makebox[0pt][r]{\scriptsize 46\hspace{1em}}}%
\@y{\@s{0}%
 The server\mbox{'}s term number.
}%
\@xx{}%
\@x{\makebox[0pt][r]{\scriptsize 47\hspace{1em}} {\VARIABLE} currentTerm}%
\@x{\makebox[0pt][r]{\scriptsize 48\hspace{1em}}}%
\@y{\@s{0}%
 The server\mbox{'}s state (Follower, \ensuremath{Candidate}, or
 \ensuremath{Leader}).
}%
\@xx{}%
\@x{\makebox[0pt][r]{\scriptsize 49\hspace{1em}} {\VARIABLE} state}%
\@x{\makebox[0pt][r]{\scriptsize 50\hspace{1em}}}%
\@y{\@s{0}%
 The candidate the server voted for in its current term, or
}%
\@xx{}%
\@x{\makebox[0pt][r]{\scriptsize 51\hspace{1em}}}%
\@y{\@s{0}%
 Nil if it hasn\mbox{'}t voted for any.
}%
\@xx{}%
\@x{\makebox[0pt][r]{\scriptsize 52\hspace{1em}} {\VARIABLE} votedFor}%
 \@x{\makebox[0pt][r]{\scriptsize 53\hspace{1em}} serverVars \.{\defeq}
 {\langle} currentTerm ,\, state ,\, votedFor {\rangle}}%
\@pvspace{8.0pt}%
\@x{\makebox[0pt][r]{\scriptsize 55\hspace{1em}}}%
\@y{\@s{0}%
 A Sequence of \ensuremath{log} entries. The index into this sequence is the
 index of the
}%
\@xx{}%
\@x{\makebox[0pt][r]{\scriptsize 56\hspace{1em}}}%
\@y{\@s{0}%
 \ensuremath{log} entry. Unfortunately, the Sequence module defines
 \ensuremath{Head(s)} as the entry
}%
\@xx{}%
\@x{\makebox[0pt][r]{\scriptsize 57\hspace{1em}}}%
\@y{\@s{0}%
 with index 1, so be careful not to use that!
}%
\@xx{}%
\@x{\makebox[0pt][r]{\scriptsize 58\hspace{1em}} {\VARIABLE} log}%
\@x{\makebox[0pt][r]{\scriptsize 59\hspace{1em}}}%
\@y{\@s{0}%
 The index of the latest entry in the \ensuremath{log} the state machine may
 apply.
}%
\@xx{}%
\@x{\makebox[0pt][r]{\scriptsize 60\hspace{1em}} {\VARIABLE} commitIndex}%
 \@x{\makebox[0pt][r]{\scriptsize 61\hspace{1em}} logVars \.{\defeq} {\langle}
 log ,\, commitIndex {\rangle}}%
\@pvspace{8.0pt}%
\@x{\makebox[0pt][r]{\scriptsize 63\hspace{1em}}}%
\@y{\@s{0}%
 The following variables are used only on candidates:
}%
\@xx{}%
\@x{\makebox[0pt][r]{\scriptsize 64\hspace{1em}}}%
\@y{\@s{0}%
 The set of servers from which the candidate has received a
 \ensuremath{RequestVote
}}%
\@xx{}%
\@x{\makebox[0pt][r]{\scriptsize 65\hspace{1em}}}%
\@y{\@s{0}%
 response in its \ensuremath{currentTerm}.
}%
\@xx{}%
\@x{\makebox[0pt][r]{\scriptsize 66\hspace{1em}} {\VARIABLE} votesResponded}%
\@x{\makebox[0pt][r]{\scriptsize 67\hspace{1em}}}%
\@y{\@s{0}%
 The set of servers from which the candidate has received a vote in its
}%
\@xx{}%
\@x{\makebox[0pt][r]{\scriptsize 68\hspace{1em}}}%
\@y{\@s{0}%
 \ensuremath{currentTerm}.
}%
\@xx{}%
\@x{\makebox[0pt][r]{\scriptsize 69\hspace{1em}} {\VARIABLE} votesGranted}%
\@x{\makebox[0pt][r]{\scriptsize 70\hspace{1em}}}%
\@y{\@s{0}%
 A history variable used in the proof. This would not be present in an
}%
\@xx{}%
\@x{\makebox[0pt][r]{\scriptsize 71\hspace{1em}}}%
\@y{\@s{0}%
 implementation.
}%
\@xx{}%
\@x{\makebox[0pt][r]{\scriptsize 72\hspace{1em}}}%
\@y{\@s{0}%
 Function from each server that voted for this candidate in its
 \ensuremath{currentTerm
}}%
\@xx{}%
\@x{\makebox[0pt][r]{\scriptsize 73\hspace{1em}}}%
\@y{\@s{0}%
 to that voter\mbox{'}s \ensuremath{log}.
}%
\@xx{}%
\@x{\makebox[0pt][r]{\scriptsize 74\hspace{1em}} {\VARIABLE} voterLog}%
 \@x{\makebox[0pt][r]{\scriptsize 75\hspace{1em}} candidateVars \.{\defeq}
 {\langle} votesResponded ,\, votesGranted ,\, voterLog {\rangle}}%
\@pvspace{8.0pt}%
\@x{\makebox[0pt][r]{\scriptsize 77\hspace{1em}}}%
\@y{\@s{0}%
 The following variables are used only on leaders:
}%
\@xx{}%
\@x{\makebox[0pt][r]{\scriptsize 78\hspace{1em}}}%
\@y{\@s{0}%
 The next entry to send to each follower.
}%
\@xx{}%
\@x{\makebox[0pt][r]{\scriptsize 79\hspace{1em}} {\VARIABLE} nextIndex}%
\@x{\makebox[0pt][r]{\scriptsize 80\hspace{1em}}}%
\@y{\@s{0}%
 The latest entry that each follower has acknowledged is the same as the
}%
\@xx{}%
\@x{\makebox[0pt][r]{\scriptsize 81\hspace{1em}}}%
\@y{\@s{0}%
 leader\mbox{'}s. This is used to calculate \ensuremath{commitIndex} on the
 leader.
}%
\@xx{}%
\@x{\makebox[0pt][r]{\scriptsize 82\hspace{1em}} {\VARIABLE} matchIndex}%
 \@x{\makebox[0pt][r]{\scriptsize 83\hspace{1em}} leaderVars \.{\defeq}
 {\langle} nextIndex ,\, matchIndex ,\, elections {\rangle}}%
\@pvspace{8.0pt}%
\@x{\makebox[0pt][r]{\scriptsize 85\hspace{1em}}}%
\@y{\@s{0}%
 End of per server variables.
}%
\@xx{}%
\@x{\makebox[0pt][r]{\scriptsize 86\hspace{1em}}}\midbar\@xx{}%
\@pvspace{8.0pt}%
\@x{\makebox[0pt][r]{\scriptsize 88\hspace{1em}}}%
\@y{\@s{0}%
 All variables; used for stuttering (asserting state hasn\mbox{'}t changed).
}%
\@xx{}%
 \@x{\makebox[0pt][r]{\scriptsize 89\hspace{1em}} vars \.{\defeq} {\langle}
 messages ,\, allLogs ,\, serverVars ,\, candidateVars ,\, leaderVars ,\,
 logVars {\rangle}}%
\@pvspace{8.0pt}%
\@x{\makebox[0pt][r]{\scriptsize 91\hspace{1em}}}\midbar\@xx{}%
\@x{\makebox[0pt][r]{\scriptsize 92\hspace{1em}}}%
\@y{\@s{0}%
 Helpers
}%
\@xx{}%
\@pvspace{8.0pt}%
\@x{\makebox[0pt][r]{\scriptsize 94\hspace{1em}}}%
\@y{\@s{0}%
 The set of all quorums. This just calculates simple majorities, but the only
}%
\@xx{}%
\@x{\makebox[0pt][r]{\scriptsize 95\hspace{1em}}}%
\@y{\@s{0}%
 important property is that every quorum overlaps with every other.
}%
\@xx{}%
 \@x{\makebox[0pt][r]{\scriptsize 96\hspace{1em}} Quorum \.{\defeq} \{ i
 \.{\in} {\SUBSET} ( Server ) \.{:} Cardinality ( i ) \.{*} 2 \.{>}
 Cardinality ( Server ) \}}%
\@pvspace{8.0pt}%
\@x{\makebox[0pt][r]{\scriptsize 98\hspace{1em}}}%
\@y{\@s{0}%
 The term of the last entry in a \ensuremath{log}, or 0 if the
 \ensuremath{log} is empty.
}%
\@xx{}%
 \@x{\makebox[0pt][r]{\scriptsize 99\hspace{1em}} LastTerm ( xlog ) \.{\defeq}
 {\IF} Len ( xlog ) \.{=} 0 \.{\THEN} 0 \.{\ELSE} xlog [ Len ( xlog ) ] .
 term}%
\@pvspace{8.0pt}%
\@x{\makebox[0pt][r]{\scriptsize 101\hspace{1em}}}%
\@y{\@s{0}%
 Helper for \ensuremath{Send} and \ensuremath{Reply}. Given a message
 \ensuremath{m} and bag of messages, return a
}%
\@xx{}%
\@x{\makebox[0pt][r]{\scriptsize 102\hspace{1em}}}%
\@y{\@s{0}%
 new bag of messages with one more \ensuremath{m} in it.
}%
\@xx{}%
 \@x{\makebox[0pt][r]{\scriptsize 103\hspace{1em}} WithMessage ( m ,\, msgs )
 \.{\defeq}}%
 \@x{\makebox[0pt][r]{\scriptsize 104\hspace{1em}}\@s{18.03} {\IF} m \.{\in}
 {\DOMAIN} msgs \.{\THEN}}%
 \@x{\makebox[0pt][r]{\scriptsize 105\hspace{1em}}\@s{35.86} [ msgs {\EXCEPT}
 {\bang} [ m ] \.{=} msgs [ m ] \.{+} 1 ]}%
\@x{\makebox[0pt][r]{\scriptsize 106\hspace{1em}}\@s{18.03} \.{\ELSE}}%
 \@x{\makebox[0pt][r]{\scriptsize 107\hspace{1em}}\@s{36.07} msgs \.{\,@@\,} (
 m \.{\colongt} 1 )}%
\@pvspace{8.0pt}%
\@x{\makebox[0pt][r]{\scriptsize 109\hspace{1em}}}%
\@y{\@s{0}%
 Helper for \ensuremath{Discard} and \ensuremath{Reply}. Given a message
 \ensuremath{m} and bag of messages, return
}%
\@xx{}%
\@x{\makebox[0pt][r]{\scriptsize 110\hspace{1em}}}%
\@y{\@s{0}%
 a new bag of messages with one less \ensuremath{m} in it.
}%
\@xx{}%
 \@x{\makebox[0pt][r]{\scriptsize 111\hspace{1em}} WithoutMessage ( m ,\, msgs
 ) \.{\defeq}}%
 \@x{\makebox[0pt][r]{\scriptsize 112\hspace{1em}}\@s{18.03} {\IF} m \.{\in}
 {\DOMAIN} msgs \.{\THEN}}%
 \@x{\makebox[0pt][r]{\scriptsize 113\hspace{1em}}\@s{35.86} [ msgs {\EXCEPT}
 {\bang} [ m ] \.{=} msgs [ m ] \.{-} 1 ]}%
\@x{\makebox[0pt][r]{\scriptsize 114\hspace{1em}}\@s{18.03} \.{\ELSE}}%
\@x{\makebox[0pt][r]{\scriptsize 115\hspace{1em}}\@s{36.07} msgs}%
\@pvspace{8.0pt}%
\@x{\makebox[0pt][r]{\scriptsize 117\hspace{1em}}}%
\@y{\@s{0}%
 Add a message to the bag of messages.
}%
\@xx{}%
 \@x{\makebox[0pt][r]{\scriptsize 118\hspace{1em}} Send ( m ) \.{\defeq}
 messages \.{'} \.{=} WithMessage ( m ,\, messages )}%
\@pvspace{8.0pt}%
\@x{\makebox[0pt][r]{\scriptsize 120\hspace{1em}}}%
\@y{\@s{0}%
 Remove a message from the bag of messages. Used when a server is done
}%
\@xx{}%
\@x{\makebox[0pt][r]{\scriptsize 121\hspace{1em}}}%
\@y{\@s{0}%
 processing a message.
}%
\@xx{}%
 \@x{\makebox[0pt][r]{\scriptsize 122\hspace{1em}} Discard ( m ) \.{\defeq}
 messages \.{'} \.{=} WithoutMessage ( m ,\, messages )}%
\@pvspace{8.0pt}%
\@x{\makebox[0pt][r]{\scriptsize 124\hspace{1em}}}%
\@y{\@s{0}%
 Combination of \ensuremath{Send} and \ensuremath{Discard
}}%
\@xx{}%
 \@x{\makebox[0pt][r]{\scriptsize 125\hspace{1em}} Reply ( response ,\,
 request ) \.{\defeq}}%
 \@x{\makebox[0pt][r]{\scriptsize 126\hspace{1em}}\@s{18.03} messages \.{'}
 \.{=} WithoutMessage ( request ,\, WithMessage ( response ,\, messages ) )}%
\@pvspace{8.0pt}%
\@x{\makebox[0pt][r]{\scriptsize 128\hspace{1em}}}%
\@y{\@s{0}%
 Return the minimum value from a set, or undefined if the set is empty.
}%
\@xx{}%
 \@x{\makebox[0pt][r]{\scriptsize 129\hspace{1em}} Min ( s ) \.{\defeq}
 {\CHOOSE} x \.{\in} s \.{:} \A\, y \.{\in} s \.{:} x \.{\leq} y}%
\@x{\makebox[0pt][r]{\scriptsize 130\hspace{1em}}}%
\@y{\@s{0}%
 Return the maximum value from a set, or undefined if the set is empty.
}%
\@xx{}%
 \@x{\makebox[0pt][r]{\scriptsize 131\hspace{1em}} Max ( s ) \.{\defeq}
 {\CHOOSE} x \.{\in} s \.{:} \A\, y \.{\in} s \.{:} x \.{\geq} y}%
\@pvspace{8.0pt}%
\@x{\makebox[0pt][r]{\scriptsize 133\hspace{1em}}}\midbar\@xx{}%
\@x{\makebox[0pt][r]{\scriptsize 134\hspace{1em}}}%
\@y{\@s{0}%
 Define initial values for all variables
}%
\@xx{}%
\@pvspace{8.0pt}%
 \@x{\makebox[0pt][r]{\scriptsize 136\hspace{1em}} InitHistoryVars \.{\defeq}
 \.{\land} elections \.{=} \{ \}}%
 \@x{\makebox[0pt][r]{\scriptsize 137\hspace{1em}}\@s{96.09} \.{\land}
 allLogs\@s{12.27} \.{=} \{ \}}%
 \@x{\makebox[0pt][r]{\scriptsize 138\hspace{1em}}\@s{96.09} \.{\land}
 voterLog\@s{0.19} \.{=} [ i \.{\in} Server \.{\mapsto} [ j \.{\in} \{ \}
 \.{\mapsto} {\langle} {\rangle} ] ]}%
 \@x{\makebox[0pt][r]{\scriptsize 139\hspace{1em}} InitServerVars \.{\defeq}
 \.{\land} currentTerm \.{=} [ i \.{\in} Server \.{\mapsto} 1 ]}%
 \@x{\makebox[0pt][r]{\scriptsize 140\hspace{1em}}\@s{91.44} \.{\land}
 state\@s{37.65} \.{=} [ i \.{\in} Server \.{\mapsto} Follower ]}%
 \@x{\makebox[0pt][r]{\scriptsize 141\hspace{1em}}\@s{91.44} \.{\land}
 votedFor\@s{18.81} \.{=} [ i \.{\in} Server \.{\mapsto} Nil ]}%
 \@x{\makebox[0pt][r]{\scriptsize 142\hspace{1em}} InitCandidateVars
 \.{\defeq} \.{\land} votesResponded \.{=} [ i \.{\in} Server \.{\mapsto} \{
 \} ]}%
 \@x{\makebox[0pt][r]{\scriptsize 143\hspace{1em}}\@s{109.35} \.{\land}
 votesGranted\@s{11.40} \.{=} [ i \.{\in} Server \.{\mapsto} \{ \} ]}%
\@x{\makebox[0pt][r]{\scriptsize 144\hspace{1em}}}%
\@y{\@s{0}%
 The values \ensuremath{nextIndex[i][i]} and \ensuremath{matchIndex[i][i]}
 are never read, since the
}%
\@xx{}%
\@x{\makebox[0pt][r]{\scriptsize 145\hspace{1em}}}%
\@y{\@s{0}%
 leader does not send itself messages. It\mbox{'}s still easier to include
 these
}%
\@xx{}%
\@x{\makebox[0pt][r]{\scriptsize 146\hspace{1em}}}%
\@y{\@s{0}%
 in the functions.
}%
\@xx{}%
 \@x{\makebox[0pt][r]{\scriptsize 147\hspace{1em}} InitLeaderVars \.{\defeq}
 \.{\land} nextIndex\@s{8.91} \.{=} [ i \.{\in} Server \.{\mapsto} [ j
 \.{\in} Server \.{\mapsto} 1 ] ]}%
 \@x{\makebox[0pt][r]{\scriptsize 148\hspace{1em}}\@s{92.58} \.{\land}
 matchIndex \.{=} [ i \.{\in} Server \.{\mapsto} [ j \.{\in} Server
 \.{\mapsto} 0 ] ]}%
 \@x{\makebox[0pt][r]{\scriptsize 149\hspace{1em}} InitLogVars \.{\defeq}
 \.{\land} log\@s{53.05} \.{=} [ i \.{\in} Server \.{\mapsto} {\langle}
 {\rangle} ]}%
 \@x{\makebox[0pt][r]{\scriptsize 150\hspace{1em}}\@s{77.32} \.{\land}
 commitIndex\@s{4.50} \.{=} [ i \.{\in} Server \.{\mapsto} 0 ]}%
 \@x{\makebox[0pt][r]{\scriptsize 151\hspace{1em}} Init \.{\defeq} \.{\land}
 messages \.{=} [ m \.{\in} \{ \} \.{\mapsto} 0 ]}%
 \@x{\makebox[0pt][r]{\scriptsize 152\hspace{1em}}\@s{39.09} \.{\land}
 InitHistoryVars}%
 \@x{\makebox[0pt][r]{\scriptsize 153\hspace{1em}}\@s{39.09} \.{\land}
 InitServerVars}%
 \@x{\makebox[0pt][r]{\scriptsize 154\hspace{1em}}\@s{39.09} \.{\land}
 InitCandidateVars}%
 \@x{\makebox[0pt][r]{\scriptsize 155\hspace{1em}}\@s{39.09} \.{\land}
 InitLeaderVars}%
 \@x{\makebox[0pt][r]{\scriptsize 156\hspace{1em}}\@s{39.09} \.{\land}
 InitLogVars}%
\@pvspace{8.0pt}%
\@x{\makebox[0pt][r]{\scriptsize 158\hspace{1em}}}\midbar\@xx{}%
\@x{\makebox[0pt][r]{\scriptsize 159\hspace{1em}}}%
\@y{\@s{0}%
 Define state transitions
}%
\@xx{}%
\@pvspace{8.0pt}%
\@x{\makebox[0pt][r]{\scriptsize 161\hspace{1em}}}%
\@y{\@s{0}%
 Server \ensuremath{i} restarts from stable storage.
}%
\@xx{}%
\@x{\makebox[0pt][r]{\scriptsize 162\hspace{1em}}}%
\@y{\@s{0}%
 It loses everything but its \ensuremath{currentTerm}, \ensuremath{votedFor},
 and \ensuremath{log}.
}%
\@xx{}%
\@x{\makebox[0pt][r]{\scriptsize 163\hspace{1em}} Restart ( i ) \.{\defeq}}%
 \@x{\makebox[0pt][r]{\scriptsize 164\hspace{1em}}\@s{18.03}
 \.{\land}\@s{5.26} state \.{'}\@s{51.11} \.{=} [ state {\EXCEPT} {\bang} [ i
 ] \.{=} Follower ]}%
 \@x{\makebox[0pt][r]{\scriptsize 165\hspace{1em}}\@s{18.03}
 \.{\land}\@s{5.26} votesResponded \.{'} \.{=} [ votesResponded {\EXCEPT}
 {\bang} [ i ] \.{=} \{ \} ]}%
 \@x{\makebox[0pt][r]{\scriptsize 166\hspace{1em}}\@s{18.03}
 \.{\land}\@s{5.26} votesGranted \.{'}\@s{11.40} \.{=} [ votesGranted
 {\EXCEPT} {\bang} [ i ] \.{=} \{ \} ]}%
 \@x{\makebox[0pt][r]{\scriptsize 167\hspace{1em}}\@s{18.03}
 \.{\land}\@s{5.26} voterLog \.{'}\@s{34.64} \.{=} [ voterLog {\EXCEPT}
 {\bang} [ i ] \.{=} [ j \.{\in} \{ \} \.{\mapsto} {\langle} {\rangle} ] ]}%
 \@x{\makebox[0pt][r]{\scriptsize 168\hspace{1em}}\@s{18.03}
 \.{\land}\@s{5.26} nextIndex \.{'}\@s{28.59} \.{=} [ nextIndex {\EXCEPT}
 {\bang} [ i ] \.{=} [ j \.{\in} Server \.{\mapsto} 1 ] ]}%
 \@x{\makebox[0pt][r]{\scriptsize 169\hspace{1em}}\@s{18.03}
 \.{\land}\@s{5.26} matchIndex \.{'}\@s{19.68} \.{=} [ matchIndex {\EXCEPT}
 {\bang} [ i ] \.{=} [ j \.{\in} Server \.{\mapsto} 0 ] ]}%
 \@x{\makebox[0pt][r]{\scriptsize 170\hspace{1em}}\@s{18.03}
 \.{\land}\@s{5.26} commitIndex \.{'}\@s{11.93} \.{=} [ commitIndex {\EXCEPT}
 {\bang} [ i ] \.{=} 0 ]}%
 \@x{\makebox[0pt][r]{\scriptsize 171\hspace{1em}}\@s{18.03}
 \.{\land}\@s{5.26} {\UNCHANGED} {\langle} messages ,\, currentTerm ,\,
 votedFor ,\, log ,\, elections {\rangle}}%
\@pvspace{8.0pt}%
\@x{\makebox[0pt][r]{\scriptsize 173\hspace{1em}}}%
\@y{\@s{0}%
 Server \ensuremath{i} times out and starts a new election.
}%
\@xx{}%
 \@x{\makebox[0pt][r]{\scriptsize 174\hspace{1em}} Timeout ( i ) \.{\defeq}
 \.{\land} state [ i ] \.{\in} \{ Follower ,\, Candidate \}}%
 \@x{\makebox[0pt][r]{\scriptsize 175\hspace{1em}}\@s{74.44} \.{\land} state
 \.{'} \.{=} [ state {\EXCEPT} {\bang} [ i ] \.{=} Candidate ]}%
 \@x{\makebox[0pt][r]{\scriptsize 176\hspace{1em}}\@s{74.44} \.{\land}
 currentTerm \.{'} \.{=} [ currentTerm {\EXCEPT} {\bang} [ i ] \.{=}
 currentTerm [ i ] \.{+} 1 ]}%
\@x{\makebox[0pt][r]{\scriptsize 177\hspace{1em}}\@s{74.44}}%
\@y{\@s{0}%
 Most implementations would probably just set the local vote
}%
\@xx{}%
\@x{\makebox[0pt][r]{\scriptsize 178\hspace{1em}}\@s{74.44}}%
\@y{\@s{0}%
 atomically, but messaging localhost for it is weaker.
}%
\@xx{}%
 \@x{\makebox[0pt][r]{\scriptsize 179\hspace{1em}}\@s{74.44} \.{\land}
 votedFor \.{'} \.{=} [ votedFor {\EXCEPT} {\bang} [ i ] \.{=} Nil ]}%
 \@x{\makebox[0pt][r]{\scriptsize 180\hspace{1em}}\@s{74.44} \.{\land}
 votesResponded \.{'} \.{=} [ votesResponded {\EXCEPT} {\bang} [ i ] \.{=} \{
 \} ]}%
 \@x{\makebox[0pt][r]{\scriptsize 181\hspace{1em}}\@s{74.44} \.{\land}
 votesGranted \.{'}\@s{11.40} \.{=} [ votesGranted {\EXCEPT} {\bang} [ i ]
 \.{=} \{ \} ]}%
 \@x{\makebox[0pt][r]{\scriptsize 182\hspace{1em}}\@s{74.44} \.{\land}
 voterLog \.{'}\@s{33.04} \.{=} [ voterLog {\EXCEPT} {\bang} [ i ] \.{=} [ j
 \.{\in} \{ \} \.{\mapsto} {\langle} {\rangle} ] ]}%
 \@x{\makebox[0pt][r]{\scriptsize 183\hspace{1em}}\@s{74.44} \.{\land}
 {\UNCHANGED} {\langle} messages ,\, leaderVars ,\, logVars {\rangle}}%
\@pvspace{8.0pt}%
\@x{\makebox[0pt][r]{\scriptsize 185\hspace{1em}}}%
\@y{\@s{0}%
 Candidate \ensuremath{i} sends \ensuremath{j} a \ensuremath{RequestVote}
 request.
}%
\@xx{}%
 \@x{\makebox[0pt][r]{\scriptsize 186\hspace{1em}} RequestVote ( i ,\, j )
 \.{\defeq}}%
 \@x{\makebox[0pt][r]{\scriptsize 187\hspace{1em}}\@s{18.03} \.{\land} state [
 i ] \.{=} Candidate}%
 \@x{\makebox[0pt][r]{\scriptsize 188\hspace{1em}}\@s{18.03} \.{\land} j
 \.{\notin} votesResponded [ i ]}%
 \@x{\makebox[0pt][r]{\scriptsize 189\hspace{1em}}\@s{18.03} \.{\land} Send (
 [ mtype\@s{40.45} \.{\mapsto} RequestVoteRequest ,\,}%
 \@x{\makebox[0pt][r]{\scriptsize 190\hspace{1em}}\@s{61.58} mterm\@s{37.21}
 \.{\mapsto} currentTerm [ i ] ,\,}%
 \@x{\makebox[0pt][r]{\scriptsize 191\hspace{1em}}\@s{61.58}
 mlastLogTerm\@s{0.96} \.{\mapsto} LastTerm ( log [ i ] ) ,\,}%
 \@x{\makebox[0pt][r]{\scriptsize 192\hspace{1em}}\@s{61.58} mlastLogIndex
 \.{\mapsto} Len ( log [ i ] ) ,\,}%
 \@x{\makebox[0pt][r]{\scriptsize 193\hspace{1em}}\@s{61.58} msource\@s{29.95}
 \.{\mapsto} i ,\,}%
 \@x{\makebox[0pt][r]{\scriptsize 194\hspace{1em}}\@s{61.58} mdest\@s{40.51}
 \.{\mapsto} j ] )}%
 \@x{\makebox[0pt][r]{\scriptsize 195\hspace{1em}}\@s{18.03} \.{\land}
 {\UNCHANGED} {\langle} serverVars ,\, candidateVars ,\, leaderVars ,\,
 logVars {\rangle}}%
\@pvspace{8.0pt}%
\@x{\makebox[0pt][r]{\scriptsize 197\hspace{1em}}}%
\@y{\@s{0}%
 Leader \ensuremath{i} sends \ensuremath{j} an \ensuremath{AppendEntries}
 request containing up to 1 entry.
}%
\@xx{}%
\@x{\makebox[0pt][r]{\scriptsize 198\hspace{1em}}}%
\@y{\@s{0}%
 While implementations may want to send more than 1 at a time, this spec uses
}%
\@xx{}%
\@x{\makebox[0pt][r]{\scriptsize 199\hspace{1em}}}%
\@y{\@s{0}%
 just 1 because it minimizes atomic regions without loss of generality.
}%
\@xx{}%
 \@x{\makebox[0pt][r]{\scriptsize 200\hspace{1em}} AppendEntries ( i ,\, j )
 \.{\defeq}}%
 \@x{\makebox[0pt][r]{\scriptsize 201\hspace{1em}}\@s{18.03} \.{\land} i
 \.{\neq} j}%
 \@x{\makebox[0pt][r]{\scriptsize 202\hspace{1em}}\@s{18.03} \.{\land} state [
 i ] \.{=} Leader}%
 \@x{\makebox[0pt][r]{\scriptsize 203\hspace{1em}}\@s{18.03} \.{\land}
 \.{\LET} prevLogIndex \.{\defeq} nextIndex [ i ] [ j ] \.{-} 1}%
 \@x{\makebox[0pt][r]{\scriptsize 204\hspace{1em}}\@s{52.54} prevLogTerm
 \.{\defeq} {\IF} prevLogIndex \.{>} 0 \.{\THEN}}%
 \@x{\makebox[0pt][r]{\scriptsize 205\hspace{1em}}\@s{153.60} log [ i ] [
 prevLogIndex ] . term}%
\@x{\makebox[0pt][r]{\scriptsize 206\hspace{1em}}\@s{135.78} \.{\ELSE}}%
\@x{\makebox[0pt][r]{\scriptsize 207\hspace{1em}}\@s{153.82} 0}%
\@x{\makebox[0pt][r]{\scriptsize 208\hspace{1em}}\@s{52.54}}%
\@y{\@s{0}%
 Send up to 1 entry, constrained by the end of the \ensuremath{log}.
}%
\@xx{}%
 \@x{\makebox[0pt][r]{\scriptsize 209\hspace{1em}}\@s{52.54} lastEntry
 \.{\defeq} Min ( \{ Len ( log [ i ] ) ,\, nextIndex [ i ] [ j ] \.{+} 1 \}
 )}%
 \@x{\makebox[0pt][r]{\scriptsize 210\hspace{1em}}\@s{52.54} entries
 \.{\defeq} SubSeq ( log [ i ] ,\, nextIndex [ i ] [ j ] ,\, lastEntry )}%
 \@x{\makebox[0pt][r]{\scriptsize 211\hspace{1em}}\@s{30.20} \.{\IN} Send ( [
 mtype\@s{48.17} \.{\mapsto} AppendEntriesRequest ,\,}%
 \@x{\makebox[0pt][r]{\scriptsize 212\hspace{1em}}\@s{83.92} mterm\@s{44.94}
 \.{\mapsto} currentTerm [ i ] ,\,}%
 \@x{\makebox[0pt][r]{\scriptsize 213\hspace{1em}}\@s{83.92}
 mprevLogIndex\@s{4.50} \.{\mapsto} prevLogIndex ,\,}%
 \@x{\makebox[0pt][r]{\scriptsize 214\hspace{1em}}\@s{83.92}
 mprevLogTerm\@s{9.01} \.{\mapsto} prevLogTerm ,\,}%
 \@x{\makebox[0pt][r]{\scriptsize 215\hspace{1em}}\@s{83.92}
 mentries\@s{34.81} \.{\mapsto} entries ,\,}%
\@x{\makebox[0pt][r]{\scriptsize 216\hspace{1em}}\@s{83.92}}%
\@y{\@s{0}%
 \ensuremath{mlog} is used as a history variable for the proof.
}%
\@xx{}%
\@x{\makebox[0pt][r]{\scriptsize 217\hspace{1em}}\@s{83.92}}%
\@y{\@s{0}%
 It would not exist in a real implementation.
}%
\@xx{}%
 \@x{\makebox[0pt][r]{\scriptsize 218\hspace{1em}}\@s{83.92} mlog\@s{57.56}
 \.{\mapsto} log [ i ] ,\,}%
 \@x{\makebox[0pt][r]{\scriptsize 219\hspace{1em}}\@s{83.92}
 mcommitIndex\@s{9.02} \.{\mapsto} Min ( \{ commitIndex [ i ] ,\, lastEntry
 \} ) ,\,}%
 \@x{\makebox[0pt][r]{\scriptsize 220\hspace{1em}}\@s{83.92} msource\@s{41.06}
 \.{\mapsto} i ,\,}%
 \@x{\makebox[0pt][r]{\scriptsize 221\hspace{1em}}\@s{83.92} mdest\@s{51.62}
 \.{\mapsto} j ] )}%
 \@x{\makebox[0pt][r]{\scriptsize 222\hspace{1em}}\@s{18.03} \.{\land}
 {\UNCHANGED} {\langle} serverVars ,\, candidateVars ,\, leaderVars ,\,
 logVars {\rangle}}%
\@pvspace{8.0pt}%
\@x{\makebox[0pt][r]{\scriptsize 224\hspace{1em}}}%
\@y{\@s{0}%
 Candidate \ensuremath{i} transitions to leader.
}%
\@xx{}%
 \@x{\makebox[0pt][r]{\scriptsize 225\hspace{1em}} BecomeLeader ( i )
 \.{\defeq}}%
 \@x{\makebox[0pt][r]{\scriptsize 226\hspace{1em}}\@s{18.03} \.{\land} state [
 i ] \.{=} Candidate}%
 \@x{\makebox[0pt][r]{\scriptsize 227\hspace{1em}}\@s{18.03} \.{\land}
 votesGranted [ i ] \.{\in} Quorum}%
 \@x{\makebox[0pt][r]{\scriptsize 228\hspace{1em}}\@s{18.03} \.{\land} state
 \.{'}\@s{33.02} \.{=} [ state {\EXCEPT} {\bang} [ i ] \.{=} Leader ]}%
 \@x{\makebox[0pt][r]{\scriptsize 229\hspace{1em}}\@s{18.03} \.{\land}
 nextIndex \.{'}\@s{8.91} \.{=} [ nextIndex {\EXCEPT} {\bang} [ i ] \.{=}}%
 \@x{\makebox[0pt][r]{\scriptsize 230\hspace{1em}}\@s{120.41} [ j \.{\in}
 Server \.{\mapsto} Len ( log [ i ] ) \.{+} 1 ] ]}%
 \@x{\makebox[0pt][r]{\scriptsize 231\hspace{1em}}\@s{18.03} \.{\land}
 matchIndex \.{'} \.{=} [ matchIndex {\EXCEPT} {\bang} [ i ] \.{=}}%
 \@x{\makebox[0pt][r]{\scriptsize 232\hspace{1em}}\@s{120.41} [ j \.{\in}
 Server \.{\mapsto} 0 ] ]}%
 \@x{\makebox[0pt][r]{\scriptsize 233\hspace{1em}}\@s{18.03} \.{\land}
 elections \.{'}\@s{14.75} \.{=} elections \.{\cup}}%
 \@x{\makebox[0pt][r]{\scriptsize 234\hspace{1em}}\@s{121.88} \{ [
 eterm\@s{18.18} \.{\mapsto} currentTerm [ i ] ,\,}%
 \@x{\makebox[0pt][r]{\scriptsize 235\hspace{1em}}\@s{130.40}
 eleader\@s{11.97} \.{\mapsto} i ,\,}%
 \@x{\makebox[0pt][r]{\scriptsize 236\hspace{1em}}\@s{130.40} elog\@s{27.43}
 \.{\mapsto} log [ i ] ,\,}%
 \@x{\makebox[0pt][r]{\scriptsize 237\hspace{1em}}\@s{130.40} evotes\@s{16.59}
 \.{\mapsto} votesGranted [ i ] ,\,}%
 \@x{\makebox[0pt][r]{\scriptsize 238\hspace{1em}}\@s{130.40} evoterLog
 \.{\mapsto} voterLog [ i ] ] \}}%
 \@x{\makebox[0pt][r]{\scriptsize 239\hspace{1em}}\@s{18.03} \.{\land}
 {\UNCHANGED} {\langle} messages ,\, currentTerm ,\, votedFor ,\,
 candidateVars ,\, logVars {\rangle}}%
\@pvspace{8.0pt}%
\@x{\makebox[0pt][r]{\scriptsize 241\hspace{1em}}}%
\@y{\@s{0}%
 Leader \ensuremath{i} receives a client request to add \ensuremath{v} to the
 \ensuremath{log}.
}%
\@xx{}%
 \@x{\makebox[0pt][r]{\scriptsize 242\hspace{1em}} ClientRequest ( i ,\, v )
 \.{\defeq}}%
 \@x{\makebox[0pt][r]{\scriptsize 243\hspace{1em}}\@s{18.03} \.{\land} state [
 i ] \.{=} Leader}%
 \@x{\makebox[0pt][r]{\scriptsize 244\hspace{1em}}\@s{18.03} \.{\land}
 \.{\LET} entry \.{\defeq} [ term\@s{4.50} \.{\mapsto} currentTerm [ i ] ,\,}%
 \@x{\makebox[0pt][r]{\scriptsize 245\hspace{1em}}\@s{102.00} value\@s{2.42}
 \.{\mapsto} v ]}%
 \@x{\makebox[0pt][r]{\scriptsize 246\hspace{1em}}\@s{57.05} newLog \.{\defeq}
 Append ( log [ i ] ,\, entry )}%
 \@x{\makebox[0pt][r]{\scriptsize 247\hspace{1em}}\@s{30.20} \.{\IN}\@s{4.50}
 log \.{'} \.{=} [ log {\EXCEPT} {\bang} [ i ] \.{=} newLog ]}%
 \@x{\makebox[0pt][r]{\scriptsize 248\hspace{1em}}\@s{18.03} \.{\land}
 {\UNCHANGED} {\langle} messages ,\, serverVars ,\, candidateVars ,\,}%
 \@x{\makebox[0pt][r]{\scriptsize 249\hspace{1em}}\@s{98.61} leaderVars ,\,
 commitIndex {\rangle}}%
\@pvspace{8.0pt}%
\@x{\makebox[0pt][r]{\scriptsize 251\hspace{1em}}}%
\@y{\@s{0}%
 Leader \ensuremath{i} advances its \ensuremath{commitIndex}.
}%
\@xx{}%
\@x{\makebox[0pt][r]{\scriptsize 252\hspace{1em}}}%
\@y{\@s{0}%
 This is done as a separate step from handling \ensuremath{AppendEntries}
 responses,
}%
\@xx{}%
\@x{\makebox[0pt][r]{\scriptsize 253\hspace{1em}}}%
\@y{\@s{0}%
 in part to minimize atomic regions, and in part so that leaders of
}%
\@xx{}%
\@x{\makebox[0pt][r]{\scriptsize 254\hspace{1em}}}%
\@y{\@s{0}%
 single-server clusters are able to mark entries committed.
}%
\@xx{}%
 \@x{\makebox[0pt][r]{\scriptsize 255\hspace{1em}} AdvanceCommitIndex ( i )
 \.{\defeq}}%
 \@x{\makebox[0pt][r]{\scriptsize 256\hspace{1em}}\@s{18.03} \.{\land} state [
 i ] \.{=} Leader}%
 \@x{\makebox[0pt][r]{\scriptsize 257\hspace{1em}}\@s{18.03} \.{\land}
 \.{\LET}}%
\@y{\@s{0}%
 The set of servers that agree up through index.
}%
\@xx{}%
 \@x{\makebox[0pt][r]{\scriptsize 258\hspace{1em}}\@s{52.54} Agree ( index )
 \.{\defeq} \{ i \} \.{\cup} \{ k \.{\in} Server \.{:}}%
 \@x{\makebox[0pt][r]{\scriptsize 259\hspace{1em}}\@s{183.60} matchIndex \.{'}
 [ i ] [ k ] \.{\geq} index \}}%
\@x{\makebox[0pt][r]{\scriptsize 260\hspace{1em}}\@s{52.54}}%
\@y{\@s{0}%
 The maximum indexes for which a quorum agrees
}%
\@xx{}%
 \@x{\makebox[0pt][r]{\scriptsize 261\hspace{1em}}\@s{52.54} agreeIndexes
 \.{\defeq} \{ index \.{\in} 1 \.{\dotdot} Len ( log [ i ] ) \.{:}}%
 \@x{\makebox[0pt][r]{\scriptsize 262\hspace{1em}}\@s{157.45} Agree ( index )
 \.{\in} Quorum \}}%
\@x{\makebox[0pt][r]{\scriptsize 263\hspace{1em}}\@s{52.54}}%
\@y{\@s{0}%
 New value for \ensuremath{commitIndex\.{'}[i]
}}%
\@xx{}%
 \@x{\makebox[0pt][r]{\scriptsize 264\hspace{1em}}\@s{52.54} newCommitIndex
 \.{\defeq}}%
 \@x{\makebox[0pt][r]{\scriptsize 265\hspace{1em}}\@s{66.07} {\IF} \.{\land}
 agreeIndexes \.{\neq} \{ \}}%
 \@x{\makebox[0pt][r]{\scriptsize 266\hspace{1em}}\@s{79.38} \.{\land} log [ i
 ] [ Max ( agreeIndexes ) ] . term \.{=} currentTerm [ i ]}%
\@x{\makebox[0pt][r]{\scriptsize 267\hspace{1em}}\@s{66.07} \.{\THEN}}%
 \@x{\makebox[0pt][r]{\scriptsize 268\hspace{1em}}\@s{84.11} Max (
 agreeIndexes )}%
\@x{\makebox[0pt][r]{\scriptsize 269\hspace{1em}}\@s{66.07} \.{\ELSE}}%
 \@x{\makebox[0pt][r]{\scriptsize 270\hspace{1em}}\@s{84.11} commitIndex [ i
 ]}%
 \@x{\makebox[0pt][r]{\scriptsize 271\hspace{1em}}\@s{30.20} \.{\IN}
 commitIndex \.{'} \.{=} [ commitIndex {\EXCEPT} {\bang} [ i ] \.{=}
 newCommitIndex ]}%
 \@x{\makebox[0pt][r]{\scriptsize 272\hspace{1em}}\@s{18.03} \.{\land}
 {\UNCHANGED} {\langle} messages ,\, serverVars ,\, candidateVars ,\,
 leaderVars ,\, log {\rangle}}%
\@pvspace{8.0pt}%
\@x{\makebox[0pt][r]{\scriptsize 274\hspace{1em}}}\midbar\@xx{}%
\@x{\makebox[0pt][r]{\scriptsize 275\hspace{1em}}}%
\@y{\@s{0}%
 Message handlers
}%
\@xx{}%
\@x{\makebox[0pt][r]{\scriptsize 276\hspace{1em}}}%
\@y{\@s{0}%
 \ensuremath{i \.{=}} recipient, \ensuremath{j \.{=}} sender, \ensuremath{m
 \.{=}} message
}%
\@xx{}%
\@pvspace{8.0pt}%
\@x{\makebox[0pt][r]{\scriptsize 278\hspace{1em}}}%
\@y{\@s{0}%
 Server \ensuremath{i} receives a \ensuremath{RequestVote} request from
 server \ensuremath{j} with
}%
\@xx{}%
\@x{\makebox[0pt][r]{\scriptsize 279\hspace{1em}}}%
\@y{\@s{0}%
 \ensuremath{m.mterm \.{\leq} currentTerm[i]}.
}%
\@xx{}%
 \@x{\makebox[0pt][r]{\scriptsize 280\hspace{1em}} HandleRequestVoteRequest (
 i ,\, j ,\, m ) \.{\defeq}}%
 \@x{\makebox[0pt][r]{\scriptsize 281\hspace{1em}}\@s{18.03} \.{\LET} logOk
 \.{\defeq} \.{\lor} m . mlastLogTerm \.{>} LastTerm ( log [ i ] )}%
 \@x{\makebox[0pt][r]{\scriptsize 282\hspace{1em}}\@s{88.54} \.{\lor}
 \.{\land} m . mlastLogTerm \.{=} LastTerm ( log [ i ] )}%
 \@x{\makebox[0pt][r]{\scriptsize 283\hspace{1em}}\@s{100.71} \.{\land} m .
 mlastLogIndex \.{\geq} Len ( log [ i ] )}%
 \@x{\makebox[0pt][r]{\scriptsize 284\hspace{1em}}\@s{40.37} grant\@s{1.95}
 \.{\defeq} \.{\land} m . mterm \.{=} currentTerm [ i ]}%
\@x{\makebox[0pt][r]{\scriptsize 285\hspace{1em}}\@s{88.54} \.{\land} logOk}%
 \@x{\makebox[0pt][r]{\scriptsize 286\hspace{1em}}\@s{88.54} \.{\land}
 votedFor [ i ] \.{\in} \{ Nil ,\, j \}}%
 \@x{\makebox[0pt][r]{\scriptsize 287\hspace{1em}}\@s{18.03} \.{\IN} \.{\land}
 m . mterm \.{\leq} currentTerm [ i ]}%
 \@x{\makebox[0pt][r]{\scriptsize 288\hspace{1em}}\@s{40.37} \.{\land}
 \.{\lor} grant\@s{7.30} \.{\land} votedFor \.{'} \.{=} [ votedFor {\EXCEPT}
 {\bang} [ i ] \.{=} j ]}%
 \@x{\makebox[0pt][r]{\scriptsize 289\hspace{1em}}\@s{52.54} \.{\lor} {\lnot}
 grant \.{\land} {\UNCHANGED} votedFor}%
 \@x{\makebox[0pt][r]{\scriptsize 290\hspace{1em}}\@s{40.37} \.{\land} Reply (
 [ mtype\@s{38.58} \.{\mapsto} RequestVoteResponse ,\,}%
 \@x{\makebox[0pt][r]{\scriptsize 291\hspace{1em}}\@s{86.98} mterm\@s{35.34}
 \.{\mapsto} currentTerm [ i ] ,\,}%
 \@x{\makebox[0pt][r]{\scriptsize 292\hspace{1em}}\@s{86.98} mvoteGranted
 \.{\mapsto} grant ,\,}%
\@x{\makebox[0pt][r]{\scriptsize 293\hspace{1em}}\@s{86.98}}%
\@y{\@s{0}%
 \ensuremath{mlog} is used just for the \ensuremath{elections} history
 variable for
}%
\@xx{}%
\@x{\makebox[0pt][r]{\scriptsize 294\hspace{1em}}\@s{86.98}}%
\@y{\@s{0}%
 the proof. It would not exist in a real implementation.
}%
\@xx{}%
 \@x{\makebox[0pt][r]{\scriptsize 295\hspace{1em}}\@s{86.98} mlog\@s{39.05}
 \.{\mapsto} log [ i ] ,\,}%
 \@x{\makebox[0pt][r]{\scriptsize 296\hspace{1em}}\@s{86.98} msource\@s{22.55}
 \.{\mapsto} i ,\,}%
 \@x{\makebox[0pt][r]{\scriptsize 297\hspace{1em}}\@s{86.98} mdest\@s{33.10}
 \.{\mapsto} j ] ,\,}%
\@x{\makebox[0pt][r]{\scriptsize 298\hspace{1em}}\@s{86.98} m )}%
 \@x{\makebox[0pt][r]{\scriptsize 299\hspace{1em}}\@s{40.37} \.{\land}
 {\UNCHANGED} {\langle} state ,\, currentTerm ,\, candidateVars ,\,
 leaderVars ,\, logVars {\rangle}}%
\@pvspace{8.0pt}%
\@x{\makebox[0pt][r]{\scriptsize 301\hspace{1em}}}%
\@y{\@s{0}%
 Server \ensuremath{i} receives a \ensuremath{RequestVote} response from
 server \ensuremath{j} with
}%
\@xx{}%
\@x{\makebox[0pt][r]{\scriptsize 302\hspace{1em}}}%
\@y{\@s{0}%
 \ensuremath{m.mterm \.{=} currentTerm[i]}.
}%
\@xx{}%
 \@x{\makebox[0pt][r]{\scriptsize 303\hspace{1em}} HandleRequestVoteResponse (
 i ,\, j ,\, m ) \.{\defeq}}%
\@x{\makebox[0pt][r]{\scriptsize 304\hspace{1em}}\@s{18.03}}%
\@y{\@s{0}%
 This tallies votes even when the current state is not
 \ensuremath{Candidate}, but
}%
\@xx{}%
\@x{\makebox[0pt][r]{\scriptsize 305\hspace{1em}}\@s{18.03}}%
\@y{\@s{0}%
 they won\mbox{'}t be looked at, so it doesn\mbox{'}t matter.
}%
\@xx{}%
 \@x{\makebox[0pt][r]{\scriptsize 306\hspace{1em}}\@s{18.03} \.{\land} m .
 mterm \.{=} currentTerm [ i ]}%
 \@x{\makebox[0pt][r]{\scriptsize 307\hspace{1em}}\@s{18.03} \.{\land}
 votesResponded \.{'} \.{=} [ votesResponded {\EXCEPT} {\bang} [ i ] \.{=}}%
 \@x{\makebox[0pt][r]{\scriptsize 308\hspace{1em}}\@s{143.01} votesResponded [
 i ] \.{\cup} \{ j \} ]}%
 \@x{\makebox[0pt][r]{\scriptsize 309\hspace{1em}}\@s{18.03} \.{\land}
 \.{\lor} \.{\land} m . mvoteGranted}%
 \@x{\makebox[0pt][r]{\scriptsize 310\hspace{1em}}\@s{42.37} \.{\land}
 votesGranted \.{'} \.{=} [ votesGranted {\EXCEPT} {\bang} [ i ] \.{=}}%
 \@x{\makebox[0pt][r]{\scriptsize 311\hspace{1em}}\@s{155.93} votesGranted [ i
 ] \.{\cup} \{ j \} ]}%
 \@x{\makebox[0pt][r]{\scriptsize 312\hspace{1em}}\@s{42.37} \.{\land}
 voterLog \.{'} \.{=} [ voterLog {\EXCEPT} {\bang} [ i ] \.{=}}%
 \@x{\makebox[0pt][r]{\scriptsize 313\hspace{1em}}\@s{134.30} voterLog [ i ]
 \.{\,@@\,} ( j \.{\colongt} m . mlog ) ]}%
 \@x{\makebox[0pt][r]{\scriptsize 314\hspace{1em}}\@s{30.20} \.{\lor}
 \.{\land} {\lnot} m . mvoteGranted}%
 \@x{\makebox[0pt][r]{\scriptsize 315\hspace{1em}}\@s{42.37} \.{\land}
 {\UNCHANGED} {\langle} votesGranted ,\, voterLog {\rangle}}%
 \@x{\makebox[0pt][r]{\scriptsize 316\hspace{1em}}\@s{18.03} \.{\land} Discard
 ( m )}%
 \@x{\makebox[0pt][r]{\scriptsize 317\hspace{1em}}\@s{18.03} \.{\land}
 {\UNCHANGED} {\langle} serverVars ,\, votedFor ,\, leaderVars ,\, logVars
 {\rangle}}%
\@pvspace{8.0pt}%
\@x{\makebox[0pt][r]{\scriptsize 319\hspace{1em}}}%
\@y{\@s{0}%
 Server \ensuremath{i} receives an \ensuremath{AppendEntries} request from
 server \ensuremath{j} with
}%
\@xx{}%
\@x{\makebox[0pt][r]{\scriptsize 320\hspace{1em}}}%
\@y{\@s{0}%
 \ensuremath{m.mterm \.{\leq} currentTerm[i]}. This just handles
 \ensuremath{m.entries} of length 0 or 1, but
}%
\@xx{}%
\@x{\makebox[0pt][r]{\scriptsize 321\hspace{1em}}}%
\@y{\@s{0}%
 implementations could safely accept more by treating them the same as
}%
\@xx{}%
\@x{\makebox[0pt][r]{\scriptsize 322\hspace{1em}}}%
\@y{\@s{0}%
 multiple independent requests of 1 entry.
}%
\@xx{}%
 \@x{\makebox[0pt][r]{\scriptsize 323\hspace{1em}} HandleAppendEntriesRequest
 ( i ,\, j ,\, m ) \.{\defeq}}%
 \@x{\makebox[0pt][r]{\scriptsize 324\hspace{1em}}\@s{18.03} \.{\LET} logOk
 \.{\defeq} \.{\lor} m . mprevLogIndex \.{=} 0}%
 \@x{\makebox[0pt][r]{\scriptsize 325\hspace{1em}}\@s{88.54} \.{\lor}
 \.{\land} m . mprevLogIndex \.{>} 0}%
 \@x{\makebox[0pt][r]{\scriptsize 326\hspace{1em}}\@s{100.71} \.{\land} m .
 mprevLogIndex \.{\leq} Len ( log [ i ] )}%
 \@x{\makebox[0pt][r]{\scriptsize 327\hspace{1em}}\@s{100.71} \.{\land} m .
 mprevLogTerm \.{=} log [ i ] [ m . mprevLogIndex ] . term}%
 \@x{\makebox[0pt][r]{\scriptsize 328\hspace{1em}}\@s{18.03} \.{\IN} \.{\land}
 m . mterm \.{\leq} currentTerm [ i ]}%
 \@x{\makebox[0pt][r]{\scriptsize 329\hspace{1em}}\@s{40.37} \.{\land}
 \.{\lor} \.{\land}}%
\@y{\@s{0}%
 reject request
}%
\@xx{}%
 \@x{\makebox[0pt][r]{\scriptsize 330\hspace{1em}}\@s{76.87} \.{\lor} m .
 mterm \.{<} currentTerm [ i ]}%
 \@x{\makebox[0pt][r]{\scriptsize 331\hspace{1em}}\@s{76.87} \.{\lor}
 \.{\land} m . mterm \.{=} currentTerm [ i ]}%
 \@x{\makebox[0pt][r]{\scriptsize 332\hspace{1em}}\@s{89.04} \.{\land} state [
 i ] \.{=} Follower}%
 \@x{\makebox[0pt][r]{\scriptsize 333\hspace{1em}}\@s{89.04} \.{\land} {\lnot}
 logOk}%
 \@x{\makebox[0pt][r]{\scriptsize 334\hspace{1em}}\@s{64.71} \.{\land} Reply (
 [ mtype\@s{54.42} \.{\mapsto} AppendEntriesResponse ,\,}%
 \@x{\makebox[0pt][r]{\scriptsize 335\hspace{1em}}\@s{111.32} mterm\@s{51.18}
 \.{\mapsto} currentTerm [ i ] ,\,}%
 \@x{\makebox[0pt][r]{\scriptsize 336\hspace{1em}}\@s{111.32}
 msuccess\@s{40.07} \.{\mapsto} {\FALSE} ,\,}%
 \@x{\makebox[0pt][r]{\scriptsize 337\hspace{1em}}\@s{111.32}
 mmatchIndex\@s{18.04} \.{\mapsto} 0 ,\,}%
 \@x{\makebox[0pt][r]{\scriptsize 338\hspace{1em}}\@s{111.32}
 msource\@s{43.92} \.{\mapsto} i ,\,}%
 \@x{\makebox[0pt][r]{\scriptsize 339\hspace{1em}}\@s{111.32} mdest\@s{54.48}
 \.{\mapsto} j ] ,\,}%
\@x{\makebox[0pt][r]{\scriptsize 340\hspace{1em}}\@s{111.32} m )}%
 \@x{\makebox[0pt][r]{\scriptsize 341\hspace{1em}}\@s{64.71} \.{\land}
 {\UNCHANGED} {\langle} serverVars ,\, logVars {\rangle}}%
\@x{\makebox[0pt][r]{\scriptsize 342\hspace{1em}}\@s{52.54} \.{\lor}}%
\@y{\@s{0}%
 return to follower state
}%
\@xx{}%
 \@x{\makebox[0pt][r]{\scriptsize 343\hspace{1em}}\@s{64.71} \.{\land} m .
 mterm \.{=} currentTerm [ i ]}%
 \@x{\makebox[0pt][r]{\scriptsize 344\hspace{1em}}\@s{64.71} \.{\land} state [
 i ] \.{=} Candidate}%
 \@x{\makebox[0pt][r]{\scriptsize 345\hspace{1em}}\@s{64.71} \.{\land} state
 \.{'} \.{=} [ state {\EXCEPT} {\bang} [ i ] \.{=} Follower ]}%
 \@x{\makebox[0pt][r]{\scriptsize 346\hspace{1em}}\@s{64.71} \.{\land}
 {\UNCHANGED} {\langle} currentTerm ,\, votedFor ,\, logVars ,\, messages
 {\rangle}}%
\@x{\makebox[0pt][r]{\scriptsize 347\hspace{1em}}\@s{52.54} \.{\lor}}%
\@y{\@s{0}%
 accept request
}%
\@xx{}%
 \@x{\makebox[0pt][r]{\scriptsize 348\hspace{1em}}\@s{64.71} \.{\land} m .
 mterm \.{=} currentTerm [ i ]}%
 \@x{\makebox[0pt][r]{\scriptsize 349\hspace{1em}}\@s{64.71} \.{\land} state [
 i ] \.{=} Follower}%
\@x{\makebox[0pt][r]{\scriptsize 350\hspace{1em}}\@s{64.71} \.{\land} logOk}%
 \@x{\makebox[0pt][r]{\scriptsize 351\hspace{1em}}\@s{64.71} \.{\land}
 \.{\LET} index \.{\defeq} m . mprevLogIndex \.{+} 1}%
\@x{\makebox[0pt][r]{\scriptsize 352\hspace{1em}}\@s{76.87} \.{\IN} \.{\lor}}%
\@y{\@s{0}%
 already done with request
}%
\@xx{}%
 \@x{\makebox[0pt][r]{\scriptsize 353\hspace{1em}}\@s{115.89} \.{\land}
 \.{\lor} m . mentries \.{=} {\langle} {\rangle}}%
 \@x{\makebox[0pt][r]{\scriptsize 354\hspace{1em}}\@s{128.05} \.{\lor}
 \.{\land} Len ( log [ i ] ) \.{\geq} index}%
 \@x{\makebox[0pt][r]{\scriptsize 355\hspace{1em}}\@s{140.22} \.{\land} log [
 i ] [ index ] . term \.{=} m . mentries [ 1 ] . term}%
\@x{\makebox[0pt][r]{\scriptsize 356\hspace{1em}}\@s{128.05}}%
\@y{\@s{0}%
 This could make our \ensuremath{commitIndex} decrease (for
}%
\@xx{}%
\@x{\makebox[0pt][r]{\scriptsize 357\hspace{1em}}\@s{128.05}}%
\@y{\@s{0}%
 example if we process an old, duplicated request),
}%
\@xx{}%
\@x{\makebox[0pt][r]{\scriptsize 358\hspace{1em}}\@s{128.05}}%
\@y{\@s{0}%
 but that doesn\mbox{'}t really affect anything.
}%
\@xx{}%
 \@x{\makebox[0pt][r]{\scriptsize 359\hspace{1em}}\@s{115.89} \.{\land}
 commitIndex \.{'} \.{=} [ commitIndex {\EXCEPT} {\bang} [ i ] \.{=}}%
 \@x{\makebox[0pt][r]{\scriptsize 360\hspace{1em}}\@s{228.93} m . mcommitIndex
 ]}%
 \@x{\makebox[0pt][r]{\scriptsize 361\hspace{1em}}\@s{115.89} \.{\land} Reply
 ( [ mtype\@s{54.42} \.{\mapsto} AppendEntriesResponse ,\,}%
 \@x{\makebox[0pt][r]{\scriptsize 362\hspace{1em}}\@s{162.50} mterm\@s{51.18}
 \.{\mapsto} currentTerm [ i ] ,\,}%
 \@x{\makebox[0pt][r]{\scriptsize 363\hspace{1em}}\@s{162.50}
 msuccess\@s{40.07} \.{\mapsto} {\TRUE} ,\,}%
 \@x{\makebox[0pt][r]{\scriptsize 364\hspace{1em}}\@s{162.50}
 mmatchIndex\@s{18.03} \.{\mapsto} m . mprevLogIndex \.{+}}%
 \@x{\makebox[0pt][r]{\scriptsize 365\hspace{1em}}\@s{262.76} Len ( m .
 mentries ) ,\,}%
 \@x{\makebox[0pt][r]{\scriptsize 366\hspace{1em}}\@s{162.50}
 msource\@s{43.92} \.{\mapsto} i ,\,}%
 \@x{\makebox[0pt][r]{\scriptsize 367\hspace{1em}}\@s{162.50} mdest\@s{54.48}
 \.{\mapsto} j ] ,\,}%
\@x{\makebox[0pt][r]{\scriptsize 368\hspace{1em}}\@s{162.50} m )}%
 \@x{\makebox[0pt][r]{\scriptsize 369\hspace{1em}}\@s{115.89} \.{\land}
 {\UNCHANGED} {\langle} serverVars ,\, logVars {\rangle}}%
\@x{\makebox[0pt][r]{\scriptsize 370\hspace{1em}}\@s{99.21} \.{\lor}}%
\@y{\@s{0}%
 conflict: remove 1 entry
}%
\@xx{}%
 \@x{\makebox[0pt][r]{\scriptsize 371\hspace{1em}}\@s{115.89} \.{\land} m .
 mentries \.{\neq} {\langle} {\rangle}}%
 \@x{\makebox[0pt][r]{\scriptsize 372\hspace{1em}}\@s{115.89} \.{\land} Len (
 log [ i ] ) \.{\geq} index}%
 \@x{\makebox[0pt][r]{\scriptsize 373\hspace{1em}}\@s{115.89} \.{\land} log [
 i ] [ index ] . term\@s{14.57} \.{\neq} m . mentries [ 1 ] . term}%
 \@x{\makebox[0pt][r]{\scriptsize 374\hspace{1em}}\@s{115.89} \.{\land}
 \.{\LET} new \.{\defeq} [ index2 \.{\in} 1 \.{\dotdot} ( Len ( log [ i ] )
 \.{-} 1 ) \.{\mapsto}}%
 \@x{\makebox[0pt][r]{\scriptsize 375\hspace{1em}}\@s{211.80} log [ i ] [
 index2 ] ]}%
 \@x{\makebox[0pt][r]{\scriptsize 376\hspace{1em}}\@s{128.05} \.{\IN} log
 \.{'} \.{=} [ log {\EXCEPT} {\bang} [ i ] \.{=} new ]}%
 \@x{\makebox[0pt][r]{\scriptsize 377\hspace{1em}}\@s{115.89} \.{\land}
 {\UNCHANGED} {\langle} serverVars ,\, commitIndex ,\, messages {\rangle}}%
\@x{\makebox[0pt][r]{\scriptsize 378\hspace{1em}}\@s{99.21} \.{\lor}}%
\@y{\@s{0}%
 no conflict: append entry
}%
\@xx{}%
 \@x{\makebox[0pt][r]{\scriptsize 379\hspace{1em}}\@s{115.89} \.{\land} m .
 mentries \.{\neq} {\langle} {\rangle}}%
 \@x{\makebox[0pt][r]{\scriptsize 380\hspace{1em}}\@s{115.89} \.{\land} Len (
 log [ i ] ) \.{=} m . mprevLogIndex}%
 \@x{\makebox[0pt][r]{\scriptsize 381\hspace{1em}}\@s{115.89} \.{\land} log
 \.{'} \.{=} [ log {\EXCEPT} {\bang} [ i ] \.{=}}%
 \@x{\makebox[0pt][r]{\scriptsize 382\hspace{1em}}\@s{176.18} Append ( log [ i
 ] ,\, m . mentries [ 1 ] ) ]}%
 \@x{\makebox[0pt][r]{\scriptsize 383\hspace{1em}}\@s{115.89} \.{\land}
 {\UNCHANGED} {\langle} serverVars ,\, commitIndex ,\, messages {\rangle}}%
 \@x{\makebox[0pt][r]{\scriptsize 384\hspace{1em}}\@s{40.37} \.{\land}
 {\UNCHANGED} {\langle} candidateVars ,\, leaderVars {\rangle}}%
\@pvspace{8.0pt}%
\@x{\makebox[0pt][r]{\scriptsize 386\hspace{1em}}}%
\@y{\@s{0}%
 Server \ensuremath{i} receives an \ensuremath{AppendEntries} response from
 server \ensuremath{j} with
}%
\@xx{}%
\@x{\makebox[0pt][r]{\scriptsize 387\hspace{1em}}}%
\@y{\@s{0}%
 \ensuremath{m.mterm \.{=} currentTerm[i]}.
}%
\@xx{}%
 \@x{\makebox[0pt][r]{\scriptsize 388\hspace{1em}} HandleAppendEntriesResponse
 ( i ,\, j ,\, m ) \.{\defeq}}%
 \@x{\makebox[0pt][r]{\scriptsize 389\hspace{1em}}\@s{18.03} \.{\land} m .
 mterm \.{=} currentTerm [ i ]}%
 \@x{\makebox[0pt][r]{\scriptsize 390\hspace{1em}}\@s{18.03} \.{\land}
 \.{\lor} \.{\land} m . msuccess}%
\@y{\@s{0}%
 successful
}%
\@xx{}%
 \@x{\makebox[0pt][r]{\scriptsize 391\hspace{1em}}\@s{42.37} \.{\land}
 nextIndex \.{'}\@s{8.91} \.{=} [ nextIndex\@s{4.50} {\EXCEPT} {\bang} [ i ]
 [ j ]\@s{4.40} \.{=} m . mmatchIndex \.{+} 1 ]}%
 \@x{\makebox[0pt][r]{\scriptsize 392\hspace{1em}}\@s{42.37} \.{\land}
 matchIndex \.{'} \.{=} [ matchIndex {\EXCEPT} {\bang} [ i ] [ j ] \.{=} m .
 mmatchIndex ]}%
 \@x{\makebox[0pt][r]{\scriptsize 393\hspace{1em}}\@s{30.20} \.{\lor}
 \.{\land} {\lnot} m . msuccess}%
\@y{\@s{0}%
 not successful
}%
\@xx{}%
 \@x{\makebox[0pt][r]{\scriptsize 394\hspace{1em}}\@s{42.37} \.{\land}
 nextIndex \.{'} \.{=} [ nextIndex {\EXCEPT} {\bang} [ i ] [ j ] \.{=}}%
 \@x{\makebox[0pt][r]{\scriptsize 395\hspace{1em}}\@s{140.34} Max ( \{
 nextIndex [ i ] [ j ] \.{-} 1 ,\, 1 \} ) ]}%
 \@x{\makebox[0pt][r]{\scriptsize 396\hspace{1em}}\@s{42.37} \.{\land}
 {\UNCHANGED} {\langle} matchIndex {\rangle}}%
 \@x{\makebox[0pt][r]{\scriptsize 397\hspace{1em}}\@s{18.03} \.{\land} Discard
 ( m )}%
 \@x{\makebox[0pt][r]{\scriptsize 398\hspace{1em}}\@s{18.03} \.{\land}
 {\UNCHANGED} {\langle} serverVars ,\, candidateVars ,\, logVars ,\,
 elections {\rangle}}%
\@pvspace{8.0pt}%
\@x{\makebox[0pt][r]{\scriptsize 400\hspace{1em}}}%
\@y{\@s{0}%
 Any \ensuremath{RPC} with a newer term causes the recipient to advance its
 term first.
}%
\@xx{}%
 \@x{\makebox[0pt][r]{\scriptsize 401\hspace{1em}} UpdateTerm ( i ,\, j ,\, m
 ) \.{\defeq}}%
 \@x{\makebox[0pt][r]{\scriptsize 402\hspace{1em}}\@s{18.03} \.{\land} m .
 mterm \.{>} currentTerm [ i ]}%
 \@x{\makebox[0pt][r]{\scriptsize 403\hspace{1em}}\@s{18.03} \.{\land}
 currentTerm \.{'}\@s{13.52} \.{=} [ currentTerm {\EXCEPT} {\bang} [ i ]
 \.{=} m . mterm ]}%
 \@x{\makebox[0pt][r]{\scriptsize 404\hspace{1em}}\@s{18.03} \.{\land} state
 \.{'}\@s{51.18} \.{=} [ state\@s{27.05} {\EXCEPT} {\bang} [ i ]\@s{10.59}
 \.{=} Follower ]}%
 \@x{\makebox[0pt][r]{\scriptsize 405\hspace{1em}}\@s{18.03} \.{\land}
 votedFor \.{'}\@s{32.34} \.{=} [ votedFor\@s{13.52} {\EXCEPT} {\bang} [ i
 ]\@s{5.28} \.{=} Nil ]}%
\@x{\makebox[0pt][r]{\scriptsize 406\hspace{1em}}\@s{30.20}}%
\@y{\@s{0}%
 messages is unchanged so \ensuremath{m} can be processed further.
}%
\@xx{}%
 \@x{\makebox[0pt][r]{\scriptsize 407\hspace{1em}}\@s{18.03} \.{\land}
 {\UNCHANGED} {\langle} messages ,\, candidateVars ,\, leaderVars ,\, logVars
 {\rangle}}%
\@pvspace{8.0pt}%
\@x{\makebox[0pt][r]{\scriptsize 409\hspace{1em}}}%
\@y{\@s{0}%
 Responses with stale terms are ignored.
}%
\@xx{}%
 \@x{\makebox[0pt][r]{\scriptsize 410\hspace{1em}} DropStaleResponse ( i ,\, j
 ,\, m ) \.{\defeq}}%
 \@x{\makebox[0pt][r]{\scriptsize 411\hspace{1em}}\@s{18.03} \.{\land} m .
 mterm \.{<} currentTerm [ i ]}%
 \@x{\makebox[0pt][r]{\scriptsize 412\hspace{1em}}\@s{18.03} \.{\land} Discard
 ( m )}%
 \@x{\makebox[0pt][r]{\scriptsize 413\hspace{1em}}\@s{18.03} \.{\land}
 {\UNCHANGED} {\langle} serverVars ,\, candidateVars ,\, leaderVars ,\,
 logVars {\rangle}}%
\@pvspace{8.0pt}%
\@x{\makebox[0pt][r]{\scriptsize 415\hspace{1em}}}%
\@y{\@s{0}%
 Receive a message.
}%
\@xx{}%
\@x{\makebox[0pt][r]{\scriptsize 416\hspace{1em}} Receive ( m ) \.{\defeq}}%
 \@x{\makebox[0pt][r]{\scriptsize 417\hspace{1em}}\@s{18.03} \.{\LET}
 i\@s{0.46} \.{\defeq} m . mdest}%
 \@x{\makebox[0pt][r]{\scriptsize 418\hspace{1em}}\@s{40.37} j \.{\defeq} m .
 msource}%
\@x{\makebox[0pt][r]{\scriptsize 419\hspace{1em}}\@s{18.03} \.{\IN}}%
\@y{\@s{0}%
 Any \ensuremath{RPC} with a newer term causes the recipient to advance
}%
\@xx{}%
\@x{\makebox[0pt][r]{\scriptsize 420\hspace{1em}}\@s{40.37}}%
\@y{\@s{0}%
 its term first. Responses with stale terms are ignored.
}%
\@xx{}%
 \@x{\makebox[0pt][r]{\scriptsize 421\hspace{1em}}\@s{40.37} \.{\lor}
 UpdateTerm ( i ,\, j ,\, m )}%
 \@x{\makebox[0pt][r]{\scriptsize 422\hspace{1em}}\@s{40.37} \.{\lor}
 \.{\land} m . mtype \.{=} RequestVoteRequest}%
 \@x{\makebox[0pt][r]{\scriptsize 423\hspace{1em}}\@s{52.54} \.{\land}
 HandleRequestVoteRequest ( i ,\, j ,\, m )}%
 \@x{\makebox[0pt][r]{\scriptsize 424\hspace{1em}}\@s{40.37} \.{\lor}
 \.{\land} m . mtype \.{=} RequestVoteResponse}%
 \@x{\makebox[0pt][r]{\scriptsize 425\hspace{1em}}\@s{52.54} \.{\land}
 \.{\lor} DropStaleResponse ( i ,\, j ,\, m )}%
 \@x{\makebox[0pt][r]{\scriptsize 426\hspace{1em}}\@s{64.71} \.{\lor}
 HandleRequestVoteResponse ( i ,\, j ,\, m )}%
 \@x{\makebox[0pt][r]{\scriptsize 427\hspace{1em}}\@s{40.37} \.{\lor}
 \.{\land} m . mtype \.{=} AppendEntriesRequest}%
 \@x{\makebox[0pt][r]{\scriptsize 428\hspace{1em}}\@s{52.54} \.{\land}
 HandleAppendEntriesRequest ( i ,\, j ,\, m )}%
 \@x{\makebox[0pt][r]{\scriptsize 429\hspace{1em}}\@s{40.37} \.{\lor}
 \.{\land} m . mtype \.{=} AppendEntriesResponse}%
 \@x{\makebox[0pt][r]{\scriptsize 430\hspace{1em}}\@s{52.54} \.{\land}
 \.{\lor} DropStaleResponse ( i ,\, j ,\, m )}%
 \@x{\makebox[0pt][r]{\scriptsize 431\hspace{1em}}\@s{64.71} \.{\lor}
 HandleAppendEntriesResponse ( i ,\, j ,\, m )}%
\@pvspace{8.0pt}%
\@x{\makebox[0pt][r]{\scriptsize 433\hspace{1em}}}%
\@y{\@s{0}%
 End of message handlers.
}%
\@xx{}%
\@x{\makebox[0pt][r]{\scriptsize 434\hspace{1em}}}\midbar\@xx{}%
\@x{\makebox[0pt][r]{\scriptsize 435\hspace{1em}}}%
\@y{\@s{0}%
 Network state transitions
}%
\@xx{}%
\@pvspace{8.0pt}%
\@x{\makebox[0pt][r]{\scriptsize 437\hspace{1em}}}%
\@y{\@s{0}%
 The network duplicates a message
}%
\@xx{}%
 \@x{\makebox[0pt][r]{\scriptsize 438\hspace{1em}} DuplicateMessage ( m )
 \.{\defeq}}%
 \@x{\makebox[0pt][r]{\scriptsize 439\hspace{1em}}\@s{18.03} \.{\land} Send (
 m )}%
 \@x{\makebox[0pt][r]{\scriptsize 440\hspace{1em}}\@s{18.03} \.{\land}
 {\UNCHANGED} {\langle} serverVars ,\, candidateVars ,\, leaderVars ,\,
 logVars {\rangle}}%
\@pvspace{8.0pt}%
\@x{\makebox[0pt][r]{\scriptsize 442\hspace{1em}}}%
\@y{\@s{0}%
 The network drops a message
}%
\@xx{}%
 \@x{\makebox[0pt][r]{\scriptsize 443\hspace{1em}} DropMessage ( m )
 \.{\defeq}}%
 \@x{\makebox[0pt][r]{\scriptsize 444\hspace{1em}}\@s{18.03} \.{\land} Discard
 ( m )}%
 \@x{\makebox[0pt][r]{\scriptsize 445\hspace{1em}}\@s{18.03} \.{\land}
 {\UNCHANGED} {\langle} serverVars ,\, candidateVars ,\, leaderVars ,\,
 logVars {\rangle}}%
\@pvspace{8.0pt}%
\@x{\makebox[0pt][r]{\scriptsize 447\hspace{1em}}}\midbar\@xx{}%
\@x{\makebox[0pt][r]{\scriptsize 448\hspace{1em}}}%
\@y{\@s{0}%
 Defines how the variables may transition.
}%
\@xx{}%
 \@x{\makebox[0pt][r]{\scriptsize 449\hspace{1em}} Next \.{\defeq} \.{\land}
 \.{\lor} \E\, i \.{\in} Server \.{:} Restart ( i )}%
 \@x{\makebox[0pt][r]{\scriptsize 450\hspace{1em}}\@s{55.78} \.{\lor} \E\, i
 \.{\in} Server \.{:} Timeout ( i )}%
 \@x{\makebox[0pt][r]{\scriptsize 451\hspace{1em}}\@s{55.78} \.{\lor} \E\, i
 ,\, j \.{\in} Server \.{:} RequestVote ( i ,\, j )}%
 \@x{\makebox[0pt][r]{\scriptsize 452\hspace{1em}}\@s{55.78} \.{\lor} \E\, i
 \.{\in} Server \.{:} BecomeLeader ( i )}%
 \@x{\makebox[0pt][r]{\scriptsize 453\hspace{1em}}\@s{55.78} \.{\lor} \E\, i
 \.{\in} Server ,\, v \.{\in} Value \.{:} ClientRequest ( i ,\, v )}%
 \@x{\makebox[0pt][r]{\scriptsize 454\hspace{1em}}\@s{55.78} \.{\lor} \E\, i
 \.{\in} Server \.{:} AdvanceCommitIndex ( i )}%
 \@x{\makebox[0pt][r]{\scriptsize 455\hspace{1em}}\@s{55.78} \.{\lor} \E\, i
 ,\, j \.{\in} Server \.{:} AppendEntries ( i ,\, j )}%
 \@x{\makebox[0pt][r]{\scriptsize 456\hspace{1em}}\@s{55.78} \.{\lor} \E\, m
 \.{\in} {\DOMAIN} messages \.{:} Receive ( m )}%
 \@x{\makebox[0pt][r]{\scriptsize 457\hspace{1em}}\@s{55.78} \.{\lor} \E\, m
 \.{\in} {\DOMAIN} messages \.{:} DuplicateMessage ( m )}%
 \@x{\makebox[0pt][r]{\scriptsize 458\hspace{1em}}\@s{55.78} \.{\lor} \E\, m
 \.{\in} {\DOMAIN} messages \.{:} DropMessage ( m )}%
\@x{\makebox[0pt][r]{\scriptsize 459\hspace{1em}}\@s{55.78}}%
\@y{\@s{0}%
 History variable that tracks every \ensuremath{log} ever:
}%
\@xx{}%
 \@x{\makebox[0pt][r]{\scriptsize 460\hspace{1em}}\@s{43.61} \.{\land} allLogs
 \.{'} \.{=} allLogs \.{\cup} \{ log [ i ] \.{:} i \.{\in} Server \}}%
\@pvspace{8.0pt}%
\@x{\makebox[0pt][r]{\scriptsize 462\hspace{1em}}}%
\@y{\@s{0}%
 The specification must start with the initial state and transition according
}%
\@xx{}%
\@x{\makebox[0pt][r]{\scriptsize 463\hspace{1em}}}%
\@y{\@s{0}%
 to \ensuremath{Next}.
}%
\@xx{}%
 \@x{\makebox[0pt][r]{\scriptsize 464\hspace{1em}} Spec \.{\defeq} Init
 \.{\land} {\Box} [ Next ]_{ vars}}%
\@pvspace{8.0pt}%
\@x{\makebox[0pt][r]{\scriptsize 466\hspace{1em}}}\bottombar\@xx{}%


\renewcommand\A{\forall\ }
\renewcommand\E{\exists\ }
\renewcommand\implies{\Rightarrow}

\section{Proof}

\begin{lemma} % currentTerm[i] monotonically increases
\label{appendix:correctness:currenttermmonotonic}
Each server's $currentTerm$ monotonically increases:
\begin{tabbing}
\tab\=\+
$\A i \in Server : $ \\
\tab\=\+
$currentTerm[i] \leq currentTerm'[i]$
\end{tabbing}
\end{lemma}

\begin{proof}
This follows immediately from the specification.
\end{proof}

\begin{lemma} % at most one leader per term
\label{appendix:correctness:olpt}
There is at most one leader per term:
\begin{tabbing}
\tab\=\+
$\A e,f \in elections : $ \\
\tab\=\+
$e.eterm = f.eterm \implies e.eleader = f.eleader$
\end{tabbing}
This is the Election Safety property of
Figure~\ref{fig:basicraft:properties}.
\end{lemma}

\begin{sketch}
It takes votes from a quorum to become leader, voters may only vote
once per term, and any two quorums overlap.
\end{sketch}

\begin{proof}\ 
\begin{enumerate}
\item Consider two elections, $e$ and $f$, both members of $elections$,
where $e.eterm = f.eterm$.
\item $e.evotes \in Quroum$ and $f.evotes \in Quorum$,
since this is a necessary condition for members of $elections$.
\item Let $voter$ be an arbitrary member of $e.evotes \cap
f.evotes$. Such a member must exist since any two quorums overlap.
\item Once $voter$ casts a vote for $e.eleader$ in $e.eterm$, it can not
cast a vote for a different server in $e.eterm$ (the specification ensures
this: once it increments its $currentTerm$, it can never vote again for
the same server (Lemma~\ref{appendix:correctness:currenttermmonotonic});
and until then, it safely retains its vote information).
\item $e.eleader = f.eleader$, since $voter$ voted for $e.eleader$ and
$voter$ voted for $f.eleader$ in $e.eterm=f.eterm$.
\end{enumerate}
\end{proof}

\begin{lemma} % leader only appends
\label{appendix:correctness:leaderlogmonotonic}
A leader's log monotonically grows during its term:
\begin{tabbing}
\tab\=\+
$\A e \in elections :$ \\
\tab\tab\=\+
$currentTerm[e.leader] = e.term \implies$ \\
\tab\tab\=\+
$\A index \in 1..Len(log[e.leader]) :$ \\
\tab\tab\=\+
$log'[e.leader][index] = log[e.leader][index]$
\end{tabbing}
This is the Leader Append-Only property of
Figure~\ref{fig:basicraft:properties}.
\end{lemma}

\begin{sketch}
As a leader, server $i$ only appends to its log; $i$ won't ever get an
AppendEntries request from some other server for the same term, since
there is at most one leader per term; and $i$ rejects AppendEntries
requests for other terms until increasing its own term.
\end{sketch}

\begin{proof}\
\begin{enumerate}
\item Three variables are involved in the goal: $elections$,
$currentTerm$, and $log$. We consider the transitions that change each
of these variables in turn; otherwise, the invariant trivially holds by the
inductive hypothesis.
\item When a new election is added to $elections$ (a history variable
which maintains information about all successful elections), the $log$
of the leader is not changed in the same step ($log'[e.leader] =
log[e.leader]$), so the invariant is maintained.
\item $currentTerm[e.leader]$ monotonically increases by
Lemma~\ref{appendix:correctness:currenttermmonotonic}, so once $e.leader$ moves to a new
term, it will trivially satisfy the invariant forever after.
\item \emph{log} changes either from client requests or AppendEntries
requests:
\begin{enumerate}
\item Case: client request:
\begin{enumerate}
\item By the specification, the leader only appends an entry to its log,
which maintains the invariant.
\end{enumerate}
\item Case: AppendEntries request:
\begin{enumerate}
\item Only servers with $state[i] = Leader$ can send AppendEntries
requests for their $currentTerm$.
\item By Lemma~\ref{appendix:correctness:olpt}, $e.leader$ is the only
server which can ever be leader for $e.term$.
\item Servers don't send themselves AppendEntries requests (see
specification).
\item $e.leader$ will process no AppendEntries requests while its term
is $e.term$.
\end{enumerate}
\end{enumerate}
\end{enumerate}
\end{proof}

\begin{lemma} % <index, term> identifies log prefix
\label{appendix:correctness:itip}
An $\langle index, term \rangle$ pair identifies a log prefix:
\begin{tabbing}
\tab\=\+
$\A l,m \in allLogs : $ \\
\tab\=\+
$\A \langle index, term \rangle \in l :$ \\
\tab\=\+
$\langle index, term \rangle \in m \implies$ \\
\tab\=\+
$\A pindex \in 1..index : $\\
\tab\tab\=\+
$l[pindex] = m[pindex]$
\end{tabbing}
This is the Log Matching property of
Figure~\ref{fig:basicraft:properties}.
\end{lemma}

\begin{sketch}
Only leaders create entries, and they assign the new entries term
numbers that will never be assigned again by other leaders (there's at
most one leader per term).
Moreover, the consistency check in AppendEntries guarantees that when
followers accept new entries, they do so in a way that's consistent with
the leader's log at the time it sent the entries.
\end{sketch}

\begin{assertion}
If $p$ is a prefix of some log $l \in allLogs$, then $allLogs' = allLogs
\cup \{p\}$ maintains the invariant (the statement in the lemma).
\begin{enumerate}
\item This follows immediately from the invariant, since $p$'s entries
match $l$'s entries, and $p$ contributes no additional entries.
\end{enumerate}
\end{assertion}
\vspace{-6ex}

\begin{proof}[Proof by induction on an execution]\
\begin{enumerate}
\item Initial state: all of the servers' logs are empty, so $allLogs =
{\langle \rangle}$, and the invariant trivially holds.
\item Inductive step: logs change in one of the following ways:
\begin{enumerate}
\item Case: a leader adds one entry (client request)
\begin{enumerate}
\item By the inductive hypothesis, $log[leader] \in allLogs$.
\item The $\langle index, term \rangle$ of the new entry cannot exist
in any other entry in any log in $allLogs$,
since there's only one leader per term
(Lemma~\ref{appendix:correctness:olpt}) and leaders
only append to their logs
(Lemma~\ref{appendix:correctness:leaderlogmonotonic}).
\item Then $allLogs' = allLogs \cup \{log[leader] \cat \langle index, term
\rangle\}$ maintains the invariant.
\end{enumerate}

\item Case: a follower removes one entry (AppendEntries request $m$)
\begin{enumerate}
\item The invariant still holds, since $log'[follower]$ is a
prefix of $log[follower]$ (by the Assertion above).
\end{enumerate}

\item Case: a follower adds one entry (AppendEntries request $m$)
\begin{enumerate}
\item $m.mlog$ is a copy of the leader's log at the time the
leader created the AppendEntries request. 
\item $m.mlog \in allLogs$ by definition of $allLogs$.
\item In the two cases below, we show that $log'[follower]$ is a prefix
of $m.mlog$.
\item Case: $m.mprevLogIndex = 0$
\begin{enumerate}
\item $m.mentries$ is a prefix of $m.mlog$.
\item $log[follower]$ is empty, as a necessary condition for accepting
the request (the specification separates transitions for removing
a conflicting entry, replying when there is no longer any change to
make, and appending an entry).
\item $log'[follower] = m.mentries$ upon accepting the request,
which is a prefix of $m.mlog$.
\end{enumerate}
\item Case: $m.mprevLogIndex > 0$
\begin{enumerate}
\item 
$start \cat \langle m.mprevLogIndex, m.mprevLogTerm \rangle
\cat m.mentries$
is a prefix of $m.mlog$,
where $start$ is some (possibly empty) log prefix.
\item The follower accepts the request by assumption, so its log
contains the entry \\$\langle m.mprevLogIndex, m.mprevLogTerm \rangle$.
\item By the inductive hypothesis, $log[follower]$
contains the prefix \\
$start \cat \langle m.mprevLogIndex, m.mprevLogTerm \rangle$.
\item $log'[follower] = start \cat \langle m.mprevLogIndex,
m.mprevLogTerm \rangle \cat m.mentries$ \\
upon accepting the request, which is a prefix of $m.mlog$.
\end{enumerate}
\item  Because $log'[follower]$ is a prefix of $m.mlog$, the invariant
is maintained (by the Assertion above).
\end{enumerate}
\end{enumerate}
\end{enumerate}
\end{proof}

\begin{lemma} % follower appends entry becomes prefix
\label{appendix:correctness:followerprefix}
When a follower appends an entry to its log, its log after
the append is a prefix of the
leader's log at the time the leader sent the AppendEntries request:
\begin{tabbing}
\tab\=\+
$\A i \in Server :$ \\
\tab\tab\=\+
$state[i] \neq Leader \land Len(log'[i]) > Len(log[i]) \implies $ \\
\tab\tab\=\+
$\E m \in \domain messages : $ \\
\tab\tab\=\+
$\sland m.mtype = AppendEntriesRequest $ \\
$\sland \A index \in 1..Len(log'[i]) : $ \\
\tab\tab\=\+
$log'[i][index] = m.mlog[index]$
\end{tabbing}
This restates an argument from the proof of
Lemma~\ref{appendix:correctness:itip} that is useful in the proofs of
other lemmas. (The argument is difficult to make before
Lemma~\ref{appendix:correctness:itip}, since that lemma's inductive
hypothesis is key; however, the proof for this lemma follows easily from
Lemma~\ref{appendix:correctness:itip}.)
\end{lemma}

\begin{sketch}
The new entry that the follower appends to its log was also present in
the leader's log. Thus, by Lemma~\ref{appendix:correctness:itip}, the
follower's new log is a prefix of what was the leader's log.
\end{sketch}

\begin{proof}
Logs change in one of the following ways:
\begin{enumerate}
\item Case: a leader adds one entry (client request). This invariant
only applies to non-leaders.
\item Case: a follower removes one entry (AppendEntries request).
This invariant only affects logs that grow in length.
\item Case: a follower adds one entry (AppendEntries request $m$):
\begin{enumerate}
\item $m.mlog$ is a copy of the leader's log at the time the
leader created the AppendEntries request.
\item Thus, $m.mlog \in allLogs$.
\item $log'[i] \in allLogs$ by definition of $allLogs$.
\item $m.mentries$, the entry being added, is the last entry in
$log'[i]$. (This extends to multiple entries for implementations that
batch entries together.)
\item $m.mentries \in m.mlog$
\item By Lemma~\ref{appendix:correctness:itip},
the index and term of $m.mentries$ uniquely identifies a prefix of
$m.mlog$ equal to $log'[i]$.
\end{enumerate}
\end{enumerate}
\end{proof}

\begin{lemma} % current term >= log terms
\label{appendix:correctness:currenttermmax}
A server's current term is always at least as large as the terms in
its log:
\begin{tabbing}
\tab\=\+
$\A i \in Server :$ \\
\tab\tab\=\+
$\A \langle index, term \rangle \in log[i] : $ \\
\tab\tab\=\+
$term \leq currentTerm[i]$
\end{tabbing}
\end{lemma}

\begin{sketch}
Servers' current terms monotonically increase.
When leaders create new entries, they assign them their current term.
And when followers accept new entries from a leader, their current term
agrees with the
leader's term at the time it sent the entries.
\end{sketch}

\begin{proof}[Proof by induction on an execution]\
\begin{enumerate}
\item Initial state: all logs are empty, so the invariant trivially holds.
\item Inductive step: $currentTerm[i]$ changes:
\begin{enumerate}
\item By Lemma~\ref{appendix:correctness:currenttermmonotonic},
$currentTerm'[i] \geq currentTerm[i]$, so the invariant is maintained.
\end{enumerate}
\item Inductive step: logs change in one of the following ways:
\begin{enumerate}
\item Case: a leader adds one entry (client request):
\begin{enumerate}
\item By the inductive hypothesis, all entries in $log[i]$
have \\ $term \leq currentTerm[i]$.
\item The new entry's term is $currentTerm[i]$.
\item Thus, all entries in $log'[i]$ satisfy the invariant.
\end{enumerate}
\item Case: a follower removes one entry (AppendEntries request)
\begin{enumerate}
\item The invariant still holds, since only the length of the log
decreased.
\end{enumerate}
\item Case: a follower adds one entry (AppendEntries request $m$):
\begin{enumerate}
\item By the inductive hypothesis, when the leader created the request,
its current term was at least as large as the term of every entry in
its log: \\
$\A \langle index, term \rangle \in m.mlog : term \leq m.mterm$
\item $log'[i]$ is a prefix of $m.mlog$ by
Lemma~\ref{appendix:correctness:followerprefix}.
\item As a necessary condition for accepting the request,
$currentTerm[i] = m.mterm$.
\item Then $currentTerm[i]$ is at least as large as the term in
every entry in $log'[i]$, and the invariant is maintained.
\end{enumerate}
\end{enumerate}
\end{enumerate}
\end{proof}

\begin{lemma} % Log terms are monotonic
\label{appendix:correctness:monotoniclogterms}
The terms of entries grow monotonically in each log:
\begin{tabbing}
\tab\=\+
$\A l \in allLogs :$ \\
\tab\tab\=\+
$\A index \in 1..(Len(l) - 1) : $ \\
\tab\tab\=\+
$l[index].term \leq l[index + 1].term$
\end{tabbing}
\end{lemma}

\begin{sketch}
A leader maintains this by assigning new entries its current term, which
is always at least as large as the terms in its log.
When followers accept new entries, they are consistent with the leader's
log at the time it sent the entries.
\end{sketch}

\begin{proof}[Proof by induction on an execution]\
\begin{enumerate}
\item Initial state: all logs are empty, so the invariant holds.
\item Inductive step: logs change in one of the following ways:
\begin{enumerate}
\item Case: a leader adds one entry (client request)
\begin{enumerate}
\item The new entry's term is $currentTerm[leader]$
\item $currentTerm[leader]$ is at least as large as the term of any
entry in $log[leader]$, by Lemma~\ref{appendix:correctness:currenttermmax}.
\end{enumerate}
\item Case: a follower removes one entry (AppendEntries request)
\begin{enumerate}
\item The invariant still holds, since only the length of the log
decreased.
\end{enumerate}
\item Case: a follower adds one entry (AppendEntries request $m$)
\begin{enumerate}
\item $log'[follower]$ is a prefix of $m.mlog$
(by Lemma~\ref{appendix:correctness:followerprefix}).
\item $m.mlog \in allLogs$
\item By the inductive hypothesis, the terms in $m.mlog$ monotonically
grow, so the terms in $log'[follower]$ monotonically grow.
\end{enumerate}
\end{enumerate}
\end{enumerate}
\end{proof}

\begin{definition} % committed at term t
An entry $\langle index, term \rangle$ is \textbf{committed at term $t$} if
it is present in every leader's log following $t$:
\begin{tabbing}
\tab\=\+
$committed(t) \is \{$\=\+$\langle index, term \rangle : $ \\
\tab\=\+
        $\A election \in elections :$\\
\tab\=\+
            $election.eterm > t \implies$ \\
\tab\=\+
                $\langle index, term \rangle \in election.elog \}$
\end{tabbing}
\end{definition}

\begin{definition} % immediately committed
An entry $\langle index, term \rangle$ is \textbf{immediately committed}
if it is acknowledged by a quorum (including the leader) during $term$.
Lemma~\ref{appendix:correctness:immcommittedlemma} shows that these
entries are committed at $term$.
\begin{tabbing}
\tab\=\+
$immediatelyCommitted \is \{$\=\+$\langle index, term \rangle \in anyLog : $ \\
$\sland anyLog \in allLogs $ \\
$\sland \E leader \in Server,\ subquorum \in \SUBSET Server: $ \\
\tab\tab\=\+
    $\sland subquorum \cup \{leader\} \in Quorum$ \\
    $\sland \A i \in subquorum :$ \\
\tab\tab\=\+
        $\E m \in messages :$ \\
\tab\=\+
             $\sland m.mtype = AppendEntriesResponse$ \\
             $\sland m.msource = i$ \\
             $\sland m.mdest = leader$ \\
             $\sland m.mterm = term$ \\
             $\sland m.mmatchIndex \geq index\}$ \\
\end{tabbing}
\end{definition}

\begin{lemma} % Immediately committed entries are committed
\label{appendix:correctness:immcommittedlemma}
{\em Immediately committed} entries are $committed$:
\begin{tabbing}
\tab\=\+
$\A \langle index, term \rangle \in immediatelyCommitted : $ \\
\tab\tab $\langle index, term \rangle \in committed(term)$
\end{tabbing}
Along with Lemma~\ref{appendix:correctness:prefixcommittedlemma},
this is the Leader Completeness property of
Figure~\ref{fig:basicraft:properties}.

\end{lemma}

\begin{sketch}
See Section~\ref{basicraft:safety:argument}.
\end{sketch}

\begin{proof} \

\begin{enumerate}
\item Consider an entry $\langle index, term \rangle$ that is {\em
immediately committed}.

\item Define
\begin{tabbing}
\tab\=\+
$Contradicting \is \{$\=\+$election \in elections : $\\
\tab\=\+
$\sland election.eterm > term $ \\
$\sland \langle index, term \rangle \notin election.elog\}$
\end{tabbing}

\item Let $election$ be an element in $Contradicting$ with a minimal $term$
field. That is, \\ $\A e \in Contradicting : election.eterm \leq e.eterm$.
\\
If more than one election has the same term, choose the earliest one.
(The specification does not allow this to happen, but it is safe for a
leader to step down and become leader again in the same term.)

\item It suffices to show a contradiction,
which implies $Contradicting = \phi$.

\item Let $voter$ be any server that both votes in $election$ and
contains $\langle index, term \rangle$ in its log during $term$ (either it
acknowledges the entry as a follower or it was leader).
Such a server must exist since:
\begin{enumerate}
\item A quorum of servers voted in $election$ for it to succeed.
\item A quorum contains $\langle index, term \rangle$ in its log during
$term$, since $\langle index, term \rangle$ is immediately committed.
\item Any two quorums overlap.
\end{enumerate}

\item Let $voterLog \be election.evoterLog[voter]$, the voter's log at
the time it cast its vote.

\item The voter contains the entry when it cast its vote during
$election.eterm$. That is, \\ $\langle index, term \rangle \in voterLog$:

\begin{enumerate}
\item $\langle index, term \rangle$ was in the voter's log during
$term$.
\item The voter must have stored the entry in $term$ before voting in
$election.eterm$, since:
\begin{enumerate}
\item $election.eterm > term$.
\item The voter rejects requests with terms smaller than its current
term, and its current term monotonically increases
(Lemma~\ref{appendix:correctness:currenttermmonotonic}).
\end{enumerate}
\item The voter couldn't have removed the entry before casting its
vote:
\begin{enumerate}
\item Case: No $AppendEntriesRequest$ with $mterm < term$ removes the entry from the
voter's log, since $currentTerm[voter] \geq term$ upon storing the
entry (by Lemma~\ref{appendix:correctness:currenttermmax}),
and the voter rejects requests with terms smaller than
\\
$currentTerm[voter]$.
\item Case: No $AppendEntriesRequest$ with $mterm = term$ removes the entry from the
voter's log, since:
\begin{enumerate}
\item There is only one leader of $term$.
\item The leader of $term$ created and therefore
contains the entry (Lemma~\ref{appendix:correctness:leaderlogmonotonic}).
\item The leader would not send any conflicting
requests to $voter$ during $term$.
\end{enumerate}
\item Case: No $AppendEntriesRequest$ with $mterm > term$ removes the entry from the
voter's log, since:
\begin{enumerate}
\item Case: $mterm > election.eterm$:\\
This can't happen, since
$currentTerm[voter] > election.eterm$ would have prevented the voter from
voting in $term$.
\item Case: $mterm = election.eterm$:\\
Since there is at most one leader per term
(Lemma~\ref{appendix:correctness:olpt}), this request would have to come
from $election.eleader$ as a result of an earlier election in the same
term ($election.eterm$).
\\
Because a leader's log grows monotonically during its term (by
Lemma~\ref{appendix:correctness:leaderlogmonotonic}),
the leader could not have had $\langle index, term \rangle$ in
its log at the start of its term.
\\
Then there exists an earlier election with the same term in $Contradicting$;
this is a contradiction.
\item Case $mterm < election.eterm$:\\
The leader of $mterm$ must have contained the entry (otherwise its
election would also be $Contradicting$ but have a smaller term than
$election$, which is a contradiction).
Thus, the leader of $mterm$ could not send any conflicting entries to
the voter for this index,
nor could it send
any conflicting entries for prior indexes: that it has this entry
implies that it has the entire prefix before it
(Lemma~\ref{appendix:correctness:itip}).
\end{enumerate}
\end{enumerate}
\end{enumerate}

\item The log comparison during elections states the following,
since $voter$ granted its vote during $election$:
\begin{tabbing}
\tab\=\+
$\slor LastTerm(election.elog) > LastTerm(voterLog)$ \\
$\slor$\=\+$\sland LastTerm(election.elog) = LastTerm(voterLog)$ \\
            $\sland Len(election.elog) \geq Len(voterLog)$
\end{tabbing}

In the following two steps, we take each of these cases in turn and show a
contradiction.

\item Case: $LastTerm(election.elog) = LastTerm(voterLog)$ and \\
            $Len(election.elog) \geq Len(voterLog)$
\begin{enumerate}
\item The leader of $LastTerm(voterLog)$ monotonically grew its log
during its term (by
Lemma~\ref{appendix:correctness:leaderlogmonotonic}).
\item The same leader must have had $election.elog$ as its log at some
point, since it created the last entry.
\item Thus, $voterLog$ is a prefix of $election.elog$.
\item Then $\langle index, term \rangle \in election.elog$,
since $\langle index, term \rangle \in voterLog$.
\item But $election \in Contradicting$ implies that
$\langle index, term \rangle \notin election.elog$.
\end{enumerate}

\item Case: $LastTerm(election.elog) > LastTerm(voterLog)$
\begin{enumerate}
\item $LastTerm(voterLog) \geq term$,
since $\langle index, term \rangle \in voterLog$
and terms in logs grow monotonically (Lemma~\ref{appendix:correctness:monotoniclogterms}).

\item $election.eterm > LastTerm(election.elog)$
since servers increment their $currentTerm$ when starting an election,
and Lemma~\ref{appendix:correctness:currenttermmax} states that a server's $currentTerm$ is
at least as large as the terms in its log.

\item Let $prior$ be the election in $elections$ with $prior.eterm =
LastTerm(election.elog)$.
Such an election must exist since $LastTerm(election.elog) > 0$
and a server must win an election before creating an entry.

\item By transitivity, we now have the following inequalities:
\begin{tabbing}
\tab\=\+
$term \leq$ \\
\tab\=\+
$LastTerm(voterLog) <$ \\
\tab\=\+
$LastTerm(election.elog) = prior.eterm <$ \\
\tab\=\+
$election.eterm$
\end{tabbing}

\item $\langle index, term \rangle \in prior.elog$, since $prior \notin
Contradicting$ ($election$ was assumed to have the lowest term of any
election in $Contradicting$, and $prior.eterm < election.eterm$).
\item $prior.elog$ is a prefix of $election.elog$ since:
\begin{enumerate}
\item $prior.eleader$ creates entries with $prior.eterm$ by appending them to
its log, which monotonically grows during $prior.eterm$ from $prior.elog$.
\item Thus, any entry with term $prior.eterm$ must follow $prior.elog$
in all logs (by Lemma~\ref{appendix:correctness:itip}).
\item $LastTerm(election.elog) = prior.eterm$
\end{enumerate}
\item $\langle index, term \rangle \in election.elog$
\item This is a contradiction, since $election.elog$ was assumed to not
contain the committed entry ($election \in Contradicting$).
\end{enumerate}
\end{enumerate}
\end{proof}

\begin{definition} % prefix committed at term t
An entry $\langle index, term \rangle$ is \textbf{prefix committed at term $t$}
if there is another entry that is {\em committed at term $t$} following it in some log.
Lemma~\ref{appendix:correctness:prefixcommittedlemma} shows that these entries are {\em
committed at term $t$}.
\begin{tabbing}
\tab\=\+
$prefixCommitted(t) \is \{$\=\+$\langle index, term \rangle \in anyLog : $ \\
$\sland anyLog \in allLogs$ \\
$\sland \E \langle rindex, rterm \rangle \in anyLog : $ \\
\tab\tab\=\+
        $\sland index < rindex$ \\
        $\sland \langle rindex, rterm \rangle \in committed(t)\}$ \\
\end{tabbing}
\end{definition}


\begin{lemma}
\label{appendix:correctness:prefixcommittedlemma}
{\em Prefix committed} entries are $committed$ in the same term:
\begin{tabbing}
\tab $\A t : prefixCommitted(t) \subseteq committed(t)$
\end{tabbing}
Along with Lemma~\ref{appendix:correctness:immcommittedlemma},
this is the Leader Completeness property of
Figure~\ref{fig:basicraft:properties}.
\end{lemma}

\begin{sketch}
If an entry is committed, it identifies a prefix of a log in which every
entry is committed, since those entries will also be present in every
future leader's log.
\end{sketch}

\begin{proof} \ 

\begin{enumerate}
\item Consider an arbitrary entry
$\langle index, term \rangle \in prefixCommitted(t)$.
\item There exists an entry $\langle rindex, rterm \rangle \in
committed(t)$ following $\langle index, term \rangle$ in some log,
by definition of $prefixCommitted(t)$.
\item $\langle rindex, rterm \rangle$ uniquely identifies the log prefix
containing $\langle index, term \rangle$ 
(Lemma~\ref{appendix:correctness:itip}).
\item Every leader following $t$ contains $\langle index, term \rangle$,
since every leader following $t$ contains $\langle rindex, rterm \rangle$.
\item $\langle index, term \rangle \in committed(t)$
by definition of $committed(t)$.

\end{enumerate}
\end{proof}

\begin{theorem}
\label{appendix:correctness:onlycommittedapplied}
Servers only apply entries that are $committed$ in their current term:
\begin{tabbing}
\tab\=\+
$\A i \in Server : $ \\
\tab\=\+
$\sland commitIndex[i] \leq Len(log[i])$ \\
$\sland \A \langle index, term \rangle \in log[i] : $ \\
\tab\tab\tab\=\+
$index \leq commitIndex[i] \implies $ \\
$\tab \langle index, term \rangle \in committed(currentTerm[i])$
\end{tabbing}
This is equivalent to the State Machine Safety property of
Figure~\ref{fig:basicraft:properties}. (The bound on the commit index is
needed to strengthen the inductive hypothesis.)
\end{theorem}

\begin{sketch}
A leader only advances its $commitIndex$ to cover entries that are
immediately committed or prefix committed. Followers update their
$commitIndex$ from the leader's only when they have a prefix of the
leader's log.
\end{sketch}

\begin{proof}[Proof by induction on an execution]\
\begin{enumerate}

\item Initial state: trivially holds for empty logs (and
$commitIndex[i]$ is initialized to 0).

\item Inductive step: the set of entries committed at $currentTerm[i]$
changes:
\begin{enumerate}
\item Once an entry is committed at $currentTerm[i]$, all leaders
of subsequent terms will have the entry (by the definition of $committed$).
\item Thus, the set of committed entries at $currentTerm[i]$
monotonically grows.
\end{enumerate}

\item Inductive step: $commitIndex[i]$ changes:
\begin{enumerate}
\item When $commitIndex[i]$ decreases (if implementations allow this to
happen), the inductive hypothesis suffices
to show the invariant holds.
\item When $commitIndex[i]$ increases, it covers entries present in
$i$'s log that are committed:
\begin{enumerate}
\item Case: follower completes accepting AppendEntries request $m$:
\begin{enumerate}
\item Upon processing the request, the follower's log is a prefix of a
prior version of the leader's log, $m.mlog$ (by
Lemma~\ref{appendix:correctness:followerprefix}).
\item Every entry up through $commitIndex'[i]$ in $m.mlog$ is committed
by the inductive hypothesis (they were marked committed in the leader's
log when it sent the request).
\end{enumerate}
\item Case: leader $i$ processes AppendEntries response:
\begin{enumerate}
\item If the leader sets a new $commitIndex$, the conditions in the
specification
ensure that $commitIndex'[i] \in immediatelyCommitted$.
\item Every entry in the leader's log with index up to $commitIndex'[i]$
is prefix committed at $currentTerm[i]$.
\end{enumerate}
\end{enumerate}
\end{enumerate}

\item Inductive step: $currentTerm[i]$ changes:
\begin{enumerate}
\item By Lemma~\ref{appendix:correctness:currenttermmonotonic},
$currentTerm'[i] > currentTerm[i]$.
\item $committed(currentTerm[i]) \subseteq
committed(currentTerm'[i])$ by the definition of \\ $committed$.
\item Thus, the inductive hypothesis suffices to show the invariant
holds.
\end{enumerate}

\item Inductive step: logs change in one of the following ways:
\begin{enumerate}
\item Case: a leader adds one entry (client request):
\begin{enumerate}
\item Newly created entries are not marked committed, so the invariant
holds.
\end{enumerate}
\item Case: a follower removes one entry (AppendEntries request $m$):
\begin{enumerate}
\item Case: the removed entry was not marked committed on the follower:
\\ The inductive hypothesis suffices to show the invariant holds.
\item Case: the removed entry was marked committed on the follower:
\begin{enumerate}
\item $m.mterm = currentTerm[i]$, since the follower accepted the
request.
\item The removed entry is not in $m.mlog$, since it conflicts with the
request.
\item The removed entry is not present in $m.msource$'s log at the start
of its term (by Lemma~\ref{appendix:correctness:leaderlogmonotonic}).
\item The election for $m.mterm$ did not contain the removed entry.
\item The removed entry is not committed at $currentTerm[i]$.
\item This contradicts the inductive hypothesis; this case cannot occur.
\end{enumerate}
\end{enumerate}
\item Case: a follower adds one entry (AppendEntries request $m$):
\begin{enumerate}
\item Case: the new entry is not marked committed on the follower:\\
The inductive hypothesis suffices to show the invariant holds.
\item Case: the new entry is marked committed on the follower:\\
$commitIndex[i]$ must increase (which was already handled above).
\end{enumerate}
\end{enumerate}
\end{enumerate}
\end{proof}
